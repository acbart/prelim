\section{Reviewing Introductory Computing with Situated Learning Theory}

In the following section, we review existing literature on the many forms that introductory computer science has taken.
This is not intended to be an exhaustive look at all the ways that computer science has been taught to novices.

Throughout this section, we will refer to the SLED Framework and the MUSIC model in order to analyze existing approaches advantages and limitations.

Younger students tend to have less fully defined domains, which means that it is more important to rely on situational interests and short-term expectations of usefulness.

\subsection{The Content of Computer Science}

There is a serious, on-going debate about what novice students in Computer Science should learn, with limited consensus. 
Complicating this discussion is the bifurcation of introductory computing into CS-0 (which studies ``Computational Thinking'') and CS-1 (which studies pure ``Computer Science''). 
There are key differences in these two educational pathways that strongly influence course design.
First, CS-0 is a terminal course intended for non-majors -- there is little-to-no expectation that they will take CS-1. By comparison, CS-1 is a foundational course meant for students majoring in Computer Science. Typically, expectations are lower for students in CS-0, and less material is covered -- critically, there is a reduced emphasis on programming in favor of higher level knowledge of Computer Science.

However, it seems generally agreed that the CS-2 and CS-3 courses begin to focus on Data Structures and Algorithms.

Computer Science Curriculum 2013 \cite{CS2013} is an attempt to codify the entire breadth of an undergraduate CS education.
However, it does not attempt to dictate the exact implementation and order of material.

Traditionally, certain topics are typical -- decision, iteration, and the nature of data and control flow.

The ``How to Design Programs'' curriculum emphasizes a functional programming model, using a LISP-descended language named Racket.
The only looping mechanism that students are taught is recursion.

The dominance of the Object-Oriented Model in software engineering usually leads to a strong emphasis in introductory courses -- in fact, there is even an ``Objects First'' curriculum.

Test-Driven development

The strictest interpretation of SL Theory, as might be applied by Ben-Ari in \cite{ben2004situated}, would seem to demand that the content of an introductory course should teach exactly the skills expected in students future careers. 

Computational Thinking often strives to teach more than just the practical mechanics of programming and Computer Science.
There are a set of cognitive techniques
Debugging

\subsection{Contexts within Computer Science}

As part of the overarching goal to bring more students into Computer Science, a large number of contexts have been explored in Introductory computing. 
Starting with Seymour Papert's work with robotics and the LOGO programming environment in the 70s, instructors have tried to motivate students first experience. %TODO: CITE

There is a reciprocal relationship between contexts and content.
When students learn programming in the context of game development, they are almost necessarily learning content related to game development that may not be universal to computer science -- e.g., how graphical resources are organized and accessed within the game engine.
This content may be seen as a distraction by the instructor, or as useful, side knowledge - for example, if a student had to learn how to use a command line in order to compile their game, they would be learning an authentic skill that might not be considered part of the core content, but is nonetheless generally useful.
When evaluating a context, it is useful to consider what content it represents, and how authentic and useful it is.
The authenticity of content that is attached to a context affects the authenticity of the learning environment as a whole.

Context is where most of the motivation is supposed to come into play, especially with regards to Empowerment, Interest, and Usefulness.
Some contexts are entirely Interest-driven -- game development, for instance, is often seen as an enjoyable experience for young children.
A large number of curriculums use 
Alice and GreenFoot are both Java-based environments meant for game and animation development.
Scratch uses its own event-driven programming language.

Although some of these environments support an authentic language, the content that they mediate is by no means authentic. 
 
Interest-driven contexts

Game Development, Robotics, eTextiles
GreenFoot
Cowboy coders paper - what have they learned?

Media Computation
students won't find it realistic to their experiences
overly engaging and distracting (Seductive Details)
the entire curriculum is built around this one concept
Imagineering - http://dl.acm.org/citation.cfm?id=2493397

Simulations
Hard to design for success
NetLogo

Social Computing for Good, Data Science
Authenticity
Personalization/Choice - challenging
Difficult to bring in real-world data quickly and efficiently
	
\subsection{Facilitations within Computer Science}

Programming environments are simultaneously tools to facilitate the learning process, learning environments where students spend a considerable amount of time, and 

Introductory learning environments
Programming
Scratch, Alice
Python, Java, Racket

Blockly - bridging the gap

Classroom experience
Lecture
Lab
Online

\subsection{Assessment within Computer Science}

This is outside the scope.

