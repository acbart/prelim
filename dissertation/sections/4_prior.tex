\section{Existing work}

The foundation of this proposal rests on prior work developing the RealTimeWeb project, a software architecture framework that provides introductory programming students with an easy way to access and manipulate distributed real-time data\cite{bart-transforming}.
Real-time data is another 
This real-time data offers similar advantages to working with Big Data -- contextualized assignments and the experience of working with a novel technology.
The toolchain offers technical scaffolding for the students to gradually ease into, or even totally circumvent, some of the difficulties of distributed computing, including HTTP access, data validation, and result parsing.

At the heart of our project are carefully engineered, open-source client libraries through which students can access the data provided by real-time web services.
We provide client libraries for a number of data sources, such as business reviews from Yelp, weather forecasts from the National Weather Service, and social content from link-sharing site Reddit.com.
Each of these client libraries is in turn available for three common beginner languages: Java, Python, and Racket. 
These libraries do more than just streamline the process of accessing distributed data, however; each library is built with a persistence layer that enables the library to work without an internet connection.
Not only does this ensure that students without a solid internet connection can maintain productivity, it also simplifies developing unit tests. 
Figure \ref{fig-cla} demonstrates the architecture used in our libraries.

Our client libraries are easily available through a curated, online gallery; each library is designed to be quickly adapted to instructors' specific academic desires. 
This gallery also provides a tool for rapidly prototyping new libraries based on our framework.
As an open-source project, we encourage collaborators to explore and extend the tools that we have created.

The success of the RealTimeWeb project is a large part of the motivation for this project.
The lessons learned when integrating real-time data should transfer to this project where we seek to integrate Big Data.
We expect similar outcomes and experiences for this new project.


CORGIS
Quickly leveraging real-world data
Big Data 
	definition
	challenges
Real-time Web
Solves High Velocity Challenges



Careful consideration must be made when choosing problems and designing contexts so that the data leads to optimally authentic learning experiences.
	One of the big dangers when attempting to create meaningful context for learners is the problem of \textit{Preauthentication}: attempting to design for authenticity without sufficient knowledge of the audience. This is a problem shared by any approach to introductory material. Petraglia gives a compelling example \cite{preauthentication}:
	
\begin{quotation}
    The task of balancing a checkbook, for instance, may be an authentic task from the perspective of a 21-year-old, but we would question its authenticity from the perspective of a 5-year-old. But more to the point, even among 21-year-olds, for whom we believe the task should be authentic, there are some who will find any given lesson in personal finance irrelevant, inaccurate, or otherwise inappropriate. 
\end{quotation}
Preauthentication stems from over-generalizations and run-away assumptions.
If you attempt to reduce an entire classroom to a list of likes and dislikes, you run the risk of ignoring each individual learner's rich history and background that they will be building from. 
It is difficult to plan for and work against this ever-present danger when designing reusable assignments. 
Petraglia \cite{preauthentication} recommends that rather than attempting to design around students prior understanding, it is better to simply convince the learner of the authenticity of the problem.
But this is limiting, since it ignores the prior experiences and understanding that a student brings to their learning.
Instead, it would be better to find a middle ground where students are given flexibility while maintaining a relatively uniform experience for students.

Eve
Solves High Volume challenges
High Volume challenges:
Too large for the HD
Inconsistent experience for students
High Variety challenges:
Too many fields
Non-text/csv/json/xml data
