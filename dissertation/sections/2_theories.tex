\section{Educational Theories of Motivation and Cognition}

There are many theories of how people learn. 
While several modern learning theories are typically applied within Computer Science Education, one of the more popular is Situated Learning Theory \cite{ben2004situated}.

Cognition vs. Motivation
Situated Learning

%TODO: Expand introduction to this section

\subsection{Situated Learning Theory}

Situated Learning Theory, originally proposed by Lave and Wenger, argues that learning normally occurs as a function of the activity, context, and culture in which it is situated\cite{lave-situated}.
Therefore, tasks in the learning environment should parallel real-world tasks, in order to maximize the \textit{authenticity}.
Contextualization is key in these settings, as opposed to decontextualized (or ``inert'') settings.
The key difference is that learning is driven by the problem being solved, rather than the tools available – therefore, the problem being solved should lead directly to the tool being taught.

A critical element of these situated environments is the need for social interaction and collaboration, as learners become involved in and acculturated by ``Communities of Practice'' \cite{brown1989situated}.
Members of a CoP share purposes, tools, processes, and a general direction -- they should have a commonly recognized domain. 
This interaction occurs not only between individuals and the experts of the community (commonly represented by teachers) in an apprenticeship model, but also occurs between different learners as they adapt at uneven paces.
This communication between peers leads to growth among both individuals, especially when access to authentic experts is limited.
As the learner develops into an expert, they shift from the outside of the community's circle to the center, becoming more and more engaged -- this is the process of ``Legitimate Peripheral Participation''.

Authenticity is another crucial, recurring theme within Situated Learning Theory.
All instruction and assessment must be aligned with reality such that success in the former leads to success in the latter.
However, there is a subtle nuance here -- authenticity is a perceived trait, not an objective one.
Students derive value from their learning only if they \textit{perceive} authenticity, regardless of whether the instructor has successfully authenticated the experience.

Situated Learning Theory has been applied to the domain of Computer Science before, with mixed results. A seminal paper by Ben-Ari \cite{ben2004situated} explores its application and limitations. This paper is somewhat hasty in its application of SL Theory by taking a macro-level view -- they narrowly look to Open-Source and Industry Software Development communities as the only potential CoPs and interpret SL Theory as strictly requiring constant legitimacy, largely ignoring the possibility for gradual development of authenticity within individual courses and modules throughout a curriculum:

\begin{quote}
What I am claiming is that situated learning as presented in
their work cannot be accepted at face value, because it simply ignores the
enormous gap between the world of education and the world of high-tech
CoPs, which demand extensive knowledge of both CS subjects and
applications areas. This gap can only be bridged by classical decontextualized
teaching in high schools, colleges and universities.
\end{quote}

However, other researchers have found it a useful lens for analyzing curriculums. For instance, Guzdial and Tew \cite{guzdial2006imagineering} used the theory to innovatively explore and deal with the problem of inauthenticity within their Media Computation project. SL Theory clearly has value, but only as a function of the way that it is applied. In this preliminary proposal, I will take advantage of SL Theory as a generalized tool for exploring the topic of authenticity throughout introductory Computer Science.

\subsection{The Application of Situated Learning Theory}

The original work in Situated Learning Theory was categorically not about pedagogy or instructional design- it described how people learn and the importance of context and collaboration, but it did not recommend a particular style of classroom.
However, subsequent research by Brown \cite{brown1989situated} and others expanded the theory so that it could be applied to the design of learning experiences. These expansions often naturally dictate the use of active learning techniques, demphasizing the role of lecture in favor of collaborative, problem-based learning activities.

Choi \& Hannafin \cite{situated-cognition} describe a particularly useful, concrete framework for designing situated learning environments and experiences. Because there is no official name given to this framework, as a matter of convenience we will refer to it as the Situated Learning Environment Design Framework (SLED Framework). This framework has four key principles:

\begin{description}
	\item[Context] ``... The problem's physical and conceptual structure as well as the purpose of activity and the social milieu in which it is embedded''\cite{rogoff1984everyday}, context is driven not just by the atmosphere of the problem at hand, but also by the background and culture surrounding the problem.
	A good context enables a student to find recognizable elements and build on prior understanding, eventually being able to freely transfer their learning to new contexts.
	\item[Content] The information intending to be conveyed to the students.
	If context is the backdrop to the learning, then content might be seen as the plot.
	Naturally, context and content are deeply intertwined with each other, and its difficult to talk about one without referencing the other; in fact, content is an abstract entity that needs to be made concrete through contextualization when it is delivered to the learner.
	If the information is too abstract, than it will never connect with the learner and will not be transferable to new domain.
	However, if it is too grounded in a domain, then it will not be clear how it can be re-applied elsewhere. 
	Ultimately, content must be given in a variety of forms to maximize transfer.
	Two useful methods for building content are anchored instruction (exploring scenarios, or anchors, in the context based on the content) and cognitive apprenticeship (mediating knowledge from an expert to the novice learner in a mentoring relationship).
	\item[Facilitations] The modifications to the learning experience that support and accelerate learning.
	Facilitations provide opportunities for students to internalize what they are learning by lowering the barriers that can surround situated experiences, possibly at the cost of some amount of authenticity. 
	These modifications might be technological in nature, but they can also be pedagogical.
	Although there are many different forms that Facilitations can take, Scaffolding is one of the most common.
	Scaffolding is a form of support that is intended to extend what a learner can accomplish on their own.
	This support is required at the onset of the learning process, but is unnecessary once a sufficient threshold has been passed; during this transition, the amount of scaffolding can be tuned to the learners understanding.
	In Computer Science, for instance, students often take advantage of software libraries and frameworks to create sophisticated graphical programs that would be beyond daunting if implemented from scratch.
	\item[Assessment] The methods used to assess the learning experience and the progress of the student.
	Choi \& Hannafin gives special attention to the “teach to the test” problem, and how assessment needs to change to measure students ability to solve authentic problem (as opposed to their ability to solve the test’s specific problem), and to be able to transfer their understanding when solving different but related problems.
	It is important that assessment is measured against the individualized goals and progress of a learner, requiring that any standards used be fluid and adaptable to different learners personal situations.
	Of course, assessment should be an on-going part of the learning process, providing feedback and diagnostics.
	Ultimately, the learner should join in the process of assessment as they transition to an expert – being able to meta-cognitively self-evaluate the effectiveness of ones methods and communicate results to others are key abilities of experts. 
\end{description}

%% active learning, etc.

vs. Constructivism
Learning Theories are lenses - they do not provide all the answers, but they simply give you a way to explore and analyze

Authentication is a tricky thing 
Hazards of authentication:
Pre-authentication - Everyone loves dinosaurs, so today's lesson uses dinosaur examples. What do you mean you don't want to learn about dinosaurs?

\subsection{MUSIC Model of Academic Motivation}

Situated Learning is a theory of learning, but is not a comprehensive motivational framework -- it describes how people learn, but it is limited in explaining why people commit to learning.
Instead, it is useful to turn to the dedicated motivational models as a lens to explore why people choose to participate and excel in Computer Science.
In this preliminary proposal, we lean on the MUSIC Model of Academic Motivation as a primary framework.

The decision to use the MUSIC model was based on several criteria.
Although there are many motivational models available, few strive to be holistic models specifically developed for academics.
For example, theories like Expectancy-Value and Cognitive Evaluation Theory have a wider scope and have stemmed from other disciplines such as healthcare.
The MUSIC Model is derived from a meta-analysis of these other theories, incorporating only the academically relevant components.
Further, the MUSIC model is a tool meant for both design and evaluation, allowing it to be used in all phases of this work.
Finally, the model and its associated instrument, the MUSIC Model of Academic Motivation Inventory (MMAMI) , has been extensively validated and utilized in other educational domains, making it a reliable device\cite{jones-validity}.

The MUSIC model identifies five key constructs in motivating students \cite{jones-description}:
\begin{description}
	\item[eMpowerment:] The amount of control that a student feels that they have over their learning -- e.g., course assignments, lecture topics, etc..
	\item[Usefulness:] The expectation of the student that the material they are learning will be valuable to their short and long term goals. There is no clear delineation of the time-scale for these goals, but there is nonetheless a distinction between strategic skills that students need to be successful in careers and personal interests and the tactical skills they need to complete present-day tasks.
	\item[Success:] The student's belief in their own ability to complete assignments, projects, and other elements of a course with the investment of a reasonable, fulfilling amount of work.
	\item[Interest:] The student's perception of how the assignment appeals to situational or long-term interests. The former covers the aspects of a course related to attention, while the latter covers topics related to the fully-identified areas of focuses of the student.
	\item[Caring:] The students perception of other stakeholders' attitudes toward them. These stakeholders primarily include their instructor and classmates, but also can be extended to consider other members of their learning experience (e.g., administration, external experts, etc.).
\end{description}

Students are motivated when one or more of these constructs is sufficiently activated.
They are not all required to achieve maximal levels, and in fact that is not always desired -- it is possible, for instance, for a student to feel too empowered, and become overwhelmed by possibilities.
For some of these constructs, a careful balance is required, and it may not be possible to ever achieve a minimal level; no matter how exciting you make your lecture, you may never convince your students it is interesting, although it is possible that they will still consider it useful and stay motivated.
Much like in Situated Learning Theory, students' subjective \textit{perception} of these constructs is a defining requirement and is more important than objective reality.

The MUSIC model is often used as an organizational framework and an evaluative tool.
As the former, it is a list of factors to consider when building modules, assignments, and content of a course.
At all times, instructors can consider whether they are leveraging at least one construct to motivate their students.
As the latter, it offers both a quantified instrument (MMAMI) and a structure to anchor a qualitative investigation on.
The model has also directly been used tactically in course design: Jones describes a controlled classroom experiment to motivate students by having an experimental group reflect on how a course satisfies the constructs of the MUSIC model (e.g., prompted to answer ``How will the material presented here will be useful to you?'').
Quantitative data gathered after the experiment indicated a significant increase in motivation.
%TODO: Cite Jones paper here

\subsection{Summary}

There are many theories of learning, cognition, and motivation that provide different outlooks on educational processes, models, and artifacts. In the context of this paper, we limit ourselves to Situated Learning Theory and the MUSIC Model of Academic Motivation. 

\begin{itemize}
	\item Situated Learning Theory posits that learning inseparably occurs within contexts
	\item Active and collaborative learning experiences are superior to lectures
	\item Authenticity is a crucial component of learning and must be carefully considered.
	\item Curriculum materials can be concretely analyzed by looking to their Context, Content, Facilitations, and Assessment.
	\item The MUSIC Model posits that academic motivation is a function of a student perceptions of Empowerment, Usefulness, Success, Interest, and Caring within a learning experience.
\end{itemize}


\section{A Theory of Big Data}

\begin{wrapfigure}{r}{0.5\textwidth}
    \begin{center}
        \includegraphics[width=0.48\textwidth]{"images/3VModel"}
    \end{center}
    \vspace{-\bigskipamount}
    \caption{The 3V Model of Big Data}
    \label{fig-3v}
\end{wrapfigure}

Big data has been loosely described as quantities of information that cannot be handled with traditional methods \cite{manyika2011big}.
But ``traditional methods'' is a vague phrase that has different meanings to different learners. To a Humanities major in their first CS-0 course, the traditional method to sum a list is to use Excel. In this scenario, ``big data'' means anything that won't comfortably fit into Excel's working memory.
However, to a third-year Computer Science major, the traditional method would be to write an iterative or recursive sequential loop; being given big data forces them to explore parallel models of execution.
Clearly, ``bigness'' is a function of the learner's experience, but that is still not a solid definition.

A more precise definition is the ``3V Model'' \cite{douglas2012importance}, which posits that there are three dimensions that distinguish big data from ordinary, run-of-the-mill data:

\begin{description}
	\item[Volume:] The total quantity of the information, usually measured in bytes or number of records. However, this also extends laterally: the number of fields in the structure of the data also impacts the complexity and size. The threshold at which data becomes big is a function of the hardware and software being used -- for instance, embedded systems may consider gigabyte-sized files to be big, while modern servers might not struggle until the petabyte level.
	\item[Velocity:] The rate at which new information is added to the system. High velocity big data implies a distributed architecture, since new data must be arriving from somewhere. The dynamicity of data can vary widely across these architectures, with data updating every year, every day, or even multiple times a second.
	\item[Variety:] The format or formats of the data. Ideally, data are always distributed in a way that is readily accessible -- for instance, simple text-based formats such as CSV and JSON are widely supported, relatively lightweight, and human-readable. More sophisticated data formats for image and audio are also typically well-supported, although still more complicated. However, projects using specialized, compressed binary formats or, more dangerously, multiple formats (e.g., image archives organized with XML files), are more complex.
\end{description}

\subsection{Challenges of High Velocity Data}

It is not trivial to enable introductory students to work with high velocity data, which is necessarily distributed. Without any scaffolding, it is necessary to delay the use of such data until much later in the course. In a prior paper \cite{realtimeweb-splashe}, we outline the biggest barriers to high velocity data as a context:
\begin{description}
  \item[Access] The process of programmatically downloading and parsing a web-based resource is a non-trivial procedure requiring an understanding of both basic concepts (e.g., function calls, data transformation) and specialized web technology (e.g., the difference between GET and POST calls, building query parameters).
	\item[Non-Idempotency] Because high velocity data is constantly changing, repeated calls to the same URL endpoint can return wildly different results, even over the course of a few minutes. This makes finding errors and testing considerably harder.
	\item[Consistency] Web-based APIs are controlled and developed by independent entities, which means that changes can occur at any time with little to no notification or time for reaction. This means that students' code can become out of date even during the middle of testing their final project.
	\item[Connectivity] Although internet speeds for students on a university campus are typically stable, this does not extend to off-campus students or students that are traveling. If the internet connection is down, then students might be completely unable to make progress.
	\item[Efficiency] Even when the internet connection is stable, it might not always be fast. Requiring a round-trip to a server can greatly drag on the testing and development process, frustrating the student and decreasing the time spent learning.
\end{description}

\subsection{Challenges of High Volume Data}

In this section, we highlight some of the more challenging aspects of introducing high volume data, similar to how we previously outlined the challenges of high velocity data.
Some of these challenges are technical in nature, and some of them of a more pedagogical nature.
These challenges lead to certain design requirements that must be satisfied in any scaffolding intended to introduce high volume data.

\begin{description}
	\item[Data Transmission:] Internet connections can be difficult and inconsistent, especially for off-campus
and non-traditional students. Although most modern universities boast impressive wired connection
speeds, these speeds rarely extend off-campus. And even when internet connections are top-notch,
they can still be inadequate to serving the needs of transmitting big data collections to an entire
classroom of students.
  \item[Storage:] High Volume data is not just cumbersome to transmit, but also bothersome to store. Students may find that dealing with gigabytes of data, is an unpalatable experience.
	\item[File Management:] Although learning how to interact with files on disk is a common topic in CS-1, it is usually reserved for the end of the course, and it may never be covered in a Computational Thinking course. 
\end{description}


\subsection{Challenges of High Variety Data}

\begin{description}
\item[Inconsistency of Storage:]
\item[Inconsistency of Tools:]
\item[Distribution of Data:] A common 
\end{description}

\subsection{General Challenges of Data}

\begin{description}
\item[Intentionally secured data] Differential privacy
\item[Unintentionally secured data] Dead URLs, PDF codebooks
\item[Non-uniform topologies] High volume data offers different contexts and problems than high velocity data.
For instance, high velocity data typically lends itself to small quantities of data that are relevant to the current state of the real world -- for instance, students can walk outside and feel the current weather, which should correlate to real-time weather reports made available by a weather library.
High volume data, on the other hand, lends itself to large quantities of mostly static data -- for instance, crime reports for a long period of time.
Although high velocity data gives authentic answers in the here and now, high volume data gives authentic answers for the future through trends.
Some fields have both kinds of data available -- meteorologists generate forecasts (high velocity, low volume) by studying historical climate data (high volume, low velocity).
But some fields are not amenable to both -- digital historians typically have large stores of historical information (high volume), but it does not change quickly (low velocity).
\item[Distribution of Data:] A common method when distributing data
\end{description}