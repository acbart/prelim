\section{Problem Statement}

A report by MGI and McKinsey's Business Technology Offices declares that ``... by 2018, the United States alone could face a shortage of 140,000 to 190,000 people with deep analytical skills as well as 1.5 million managers and analysts with the know-how to use the analysis of Big Data to make effective decisions''~\cite{McKinsey} .
This gap, and the expanding recognition of ``Computational Thinking'' as a 21st century skill, increasingly requires that computation be positioned in a university's general education curriculum~\cite{wing2006}.
For instance, Virginia Tech is now formalizing this requirement through a new course (``Introduction to Computational Thinking'') that all university students will eventually be required to take~\cite{vt-vision}.
Such a course poses serious pedagogical and motivational challenges: how do we introduce Computational Thinking to students with no prior computing experience and convince them that the field can tangibly benefit their respective disciplines?



We need more students in Computer Science and with Computational skills
Describe introductory programming
CS-1,2,3: Quick one-sentence descriptions
As students progress to CS-3, motivation becomes less critical compared to self-regulation
CS-0:  "Computational Thinking"
Limited time to teach relevant concepts
Desire to engage them
In order to bring in these students, we must approach this from both the cognitive and motivational sides
Want to keep students highly motivated
Decontextualized assignments
But we also have a new set of skills to teach students
Data science
Big Data
What can technology and pedagogy do to solve this problem
In this preliminary proposal, we describe work previously done and future work


Computer Science and Computational Thinking
New audiences



Introductory programming
	CS-1,2,3
	Computational Thinking
		Wing definition
Contextualization
	Abstract
		Straw-man example of boring lists
	Constructivism
	Socio-constructivism
	Situated Learning
	Authentication
		Preauthentication
		Over-authentication (?) - I like dinosaurs, but you suck at teaching dinosaurs
Scaffolding
	Fading
Media Computation
	Pre-authentication - students won't find it realistic to their experiences
	Under-authentication - not realistic enough for the students tastes
	Over-authentication - overly engaging and distracting (Seductive Details)
	
	http://dl.acm.org/citation.cfm?id=2493397
Proposed Theory
	Retention hypothesis
	Gender Hypothesis
	Learning Hypothesis
		Gains vs. Absolute
	More-Computing Hypothesis
		Code Aacdemy, Let's Code Blacksburg
	
Prior Work
	Real-time Web
Proposed Architecture
	+Big-Data
	+Sources
	Python + Blockly
Proposed curriculum
	Usable as components or a whole
	Good for students motivated by usefulness
Work Plan


Big Data is much in the news these days, from reports of massive data
dredging by the NSA to tens of millions of credit cards stolen by
hackers from commercial databases.
Big Data has become crucial to scientific advances from understanding the
genome to predicting climate change.
It would do well for anyone these days to appreciate the capabilities
and the limits of Big Data processing.
Certainly a study of Big Data concepts aligns well with current
efforts to acquaint a broad range of students to Computational
Thinking, and it certainly would be of broad interest to most of these
students.

Leveraging the promise of Big Data offers unprecedented opportunities
for improved economic growth and enhanced innovation.
Unfortunately, computer scientists in the workforce are woefully
unequipped for this shifting paradigm.
Indeed, a report by MGI and McKinsey's Business Technology Offices
declares that ``by 2018, the United States alone could face a
shortage of 140,000 to 190,000 people with deep analytical skills as
well as 1.5 million managers and analysts with the know-how to use the
analysis of Big Data to make effective decisions.''\cite{McKinsey}
In order to close this gap, we need to do more than just improve the
education of existing students; we must draw from under-represented
and non-traditional sectors.
In particular, we need to start reaching out to non-majors through the banner of Computational Thinking.

There are two main obstacles to effectively educating students on Big
Data.
First, its representation, manipulation, and expression is
challenging, with modern curriculums and programming tools being
inadequate.
In fact, the definition of Big Data is quantities of information that
cannot be handled with traditional methods\cite{McKinsey}.
Second, postponing these topics until late in the curriculum, as is
commonly done\cite{CS2013}, fails to prepare students to solve problems in this
domain; the Computer Science Curricula 2013 recommends 10 core hours
minimum, requiring more distribution of the material over the life of
an undergraduate program.
Moreover, learning how to manage this complexity is central to
interdisciplinary work in everything from agriculture to medicine to
law, and everything in between\cite{theory-end}.
We cannot expect non-majors to progress too far through our own
subject, so therefore work must be done to cover this material in
introductory courses, as early and often as possible.

Despite the difficulties, there are serious benefits to introducing Big Data problems earlier into the curriculum.
By opening an entire new class of problems, assignments can be more varied.
By exposing students to Big Data topics earlier, their foundational understanding will be stronger and they will develop a stronger interest in this important subject.
And finally, we can introduce non-majors to useful skills that they can immediately apply to their own subjects, fostering interdisciplinary work.
But how do we make this complicated and difficult topic accessible?