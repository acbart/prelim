% Proposal to be submitted to IUSE program
% http://www.nsf.gov/funding/pgm_summ.jsp?pims_id=504976
% Due February 4, 2014

\documentclass[11pt]{article}
\usepackage{fullpage}
\usepackage{graphicx}
\usepackage{url}
\usepackage{paralist}

\begin{document}

\newcommand{\NP}{\mbox{${\cal NP}$}}
\renewcommand{\floatpagefraction}{.90} % My preference
\renewcommand{\textfraction}{.10}

\begin{center}
\Large
Title?\\
Second line of title\\

\bigskip
EArly-concept Grants for Exploratory Research (EAGER)\\

\end{center}

\newpage

\pagestyle{empty}

We request support to study the efficacy 
of and disseminate a STEM learning environment named OpenDSA.
OpenDSA is an open-source project with international collaboration.
It has the potential to fundamentally change instruction in courses on
Data Structures and Algorithms (DSA) and Formal Languages and Automata
(FLA).
Such courses play a central role in Computer Science curricula.
By combining textbook-quality content with visualization and a rich
collection of automatically assessed interactive exercises,
OpenDSA can solve the following problems with DSA and FLA courses.
(1) Students often find this material difficult to comprehend because
much of the content is about dynamic processes, such as the behavior
of algorithms and their effects over time on data structures.
Static media (textbooks) do a poor job of conveying dynamic process.
(2) Algorithm visualizations (AVs) have been demonstrated to be
pedagogically effective for presenting this material,
but adoption has been lower than documented instructor
support would indicate.
OpenDSA's complete units of instruction help to ease the adoption
problems that have plagued previous standalone AV efforts.
(3) The greatest difficulty that DSA and FLA students encounter is
lack of practice and lack of feedback about whether they understand
the material.
A typical course offers only a small number of homework
problems and test problems, whose results come only long after the
student gives an answer.
OpenDSA provides a steady stream of exercises and activities
with automated grading and immediate feedback on performance.
Students (and instructors) can better know that they are on track.

Written using HTML5 standards, OpenDSA can be used on all major
browsers, most tablets, and many smart phones.
OpenDSA's core infrastructure supports development of a wide range
of interactive courseware and exercises, along with support for data
collection used to analyze student performance.
An initial core of about one semester's worth of instruction has been
completed.

This proposal seeks to scale up OpenDSA in a number of ways.
The highly successful JFLAP software for interactive instruction on
FLA will be redeployed within the OpenDSA framework using HTML5
standards, thereby increasing access.
A wide range of colleges and universities will be involved in
disseminating OpenDSA and assessing its impact on student learning,
and OpenDSA's use in a number of innovative instructional settings
will be explored.
The OpenDSA infrastructure will be enriched, allowing instructors
to more easily tailor the materials to their specific classroom needs,
and encouraging new content contributions from these instructors.
A number of technical pedagogical experiments will be conducted,
such as measuring the effects of augmenting content with audio
narration in slideshows,
and navigation through topics with concept maps.
We will study how these materials can improve teaching in a range of
courses for which we create relevant content.
Our efforts will impact future active eTextbook projects by
demonstrating successful ways to integrate content,
interactivity, and assessment in an open-source, creative-commons
environment.
We will experiment with new models of dissemination for open-source
content in conjunction with commercial online content publishers such
as Zyante.

\textbf{Intellectual Merit:}
The main research objective of this project is to test the efficacy of
OpenDSA, and determine the circumstances under which success can be
achieved.
In particular, we study the effect on student learning of integrating
content with visualizations and a rich collection of practice
exercises with automated feedback.
We also study how using eTextbook materials affects the
evolving pedagogical approaches of instructors of DSA and FLA courses.

\textbf{Broader Impact:}
Active eTextbooks are valuable beyond Computer Science.
Prior research indicates that online instruction in many fields can
be enhanced by student interaction with well-designed exercises.
Our efforts will provide an exemplar of how collaborative,
open-sourced workflows could be used to develop
active eTextbooks for many disciplines.

\newpage
\setcounter{page}{1}
\pagestyle{plain}
\section{Active eTextbooks for DSA}
\label{ProblemSec}

Data Structures and Algorithms (DSA) play a central role in
the Computer Science curricula~\cite{CS2013},
defining the transition from learning programming to learning computer
science.
However, students often find this material difficult to
comprehend because so much of the content is about the dynamic process
of algorithms, their effects over time on data structures, and their
analysis (determining their growth rates).
Dynamic process is difficult to convey using static
presentation media such as text and images in a textbook.
During lecture, instructors typically draw on the board,
trying to illustrate dynamic processes through words and constant
changes to the diagrams.
Many students have a hard time understanding these explanations at a
detailed level or cannot reproduce the intermediate steps to
get to the final result.
Another difficulty is lack of practice with problems and exercises.
Since the best types of problems for such courses are hard to grade by
hand, students normally experience only a small number of
homework and test problems, whose results come only long after the
student gives an answer.
The dearth of feedback to students regarding whether they understand
the material compounds the difficulty of teaching and learning DSA.

CS instructors have a long tradition of using Algorithm Visualization
(AV)~\cite{NapsPanelShort,Shaffer10,Fouh:AV11}
to convey dynamic concepts.
Despite the fact that measuring pedagogical effectiveness of AVs is
not an easy task, several AV systems have led
to the improvement of learner's performance.
Such improvements most likely occurr when the level of student
engagement and interaction is
higher~\cite{MetaStudy,urquiza09,Fouh:AV11}.
AV has been used as a means both to deliver the necessary dynamic
exposition, and to increase student interaction with 
the material through interactive exercises~\cite{Malmi04}.
Surveys~\cite{NapsPanelShort,ShafferSIGCSE11} show that
instructors are positive about using AVs in theory, and students are
overwhelmingly in favor of using AVs when given the opportunity.
However, these same surveys show many impediments to AV adoption,
including finding and using good materials, and ``fitting them'' into
existing classes.
Fortunately we do have experience with one area of computer science
that has had strong AV adoption.
That is formal languages and automata (FLA), where the educational
software JFLAP~\cite{JFLAPsite} has been widely adopted in
courses around the world.

The OpenDSA project seeks to address all of the issues listed above.
OpenDSA's goal is to build complete open-source, online, configurable
eTextbooks for DSA and FLA courses.
The content is presented as a series of small modules implemented
using HTML5 technology.
Thus the modules can be viewed on any modern browser with no
additional plugins or software needed, and can even run on 
tablets and many mobile devices.

OpenDSA modules combine content in the form of text, visualizations,
and simulations with a rich variety of exercises and assessment
questions.
Since OpenDSA modules are complete units of instruction, they
are easy for instructors to use as replacements for their existing
coverage of topics (similar to adopting a new textbook) rather than
including an AV on top of their existing presentation.
Since OpenDSA's exercises are immediately assessed, with problem
instances generated at random, students gain far more practice than is
possible with normal paper textbooks.
Since the content is highly visual and interactive, students not only
get to see the dynamic aspects of the processes under study,
they also get to manipulate these dynamic aspects themselves.
Emphasizing student engagement with the material conforms to the best
practices as developed through more than a decade of research by the
AV research community~\cite{NapsPanelLong,NapsLongImpact,KorhonenWG13}.

Each module includes mechanisms for students to self-gauge how
well they have understood the concepts presented.
Self-assessment can increase learner's motivation, promote students'
ability to guide their own learning and help them internalize factors
used when judging performance~\cite{mcmillan2008,heidi09}.
We do make use of simple multiple choice and give-a-number
style questions, for which we use the Khan Academy Exercise
Infrastructure~\cite{KhanAcademyExercise} to generate individual
problem instances.
We also include many interactive exercises.
We make extensive use of ``algorithm simulation'' or ``proficiency''
exercises, as pioneered by the TRAKLA2 project~\cite{Malmi04}.
(Note that the TRAKLA2 developers from Aalto University in Helsinki are
active participants in OpenDSA, having developed the JSAV graphics
library~\cite{Karavirta:ITiCSE13,JSAVsite} and several OpenDSA exercises.
One could view OpenDSA as ``TRAKLA3''.)
In algorithm proficiency exercises, students are shown a data
structure in a graphical interface,
and must manipulate it to demonstrate knowledge of an algorithmic
process.
For example, they might show the swap operations that a given sorting
algorithm uses.
Or they might show the changes that take place when a new element is
inserted into a tree structure.
Other OpenDSA exercises make use of small simulations for algorithms
or mathematical equations to let students see the effects that result
from changing the input parameters.
Small-scale programming exercises are automatically assessed for
correctness.
These problems are similar to small homework problems traditionally
given in such a course, but which have been hard to grade.

A key feature of OpenDSA is that it is an open-source project
(hence the name).
This is significant in a number of ways.
One is that instructors are supported in configuring OpenDSA to their
needs (selecting the modules that they need for their course).
They can also revise any aspect of the material as they choose, and
contribute such changes back to the project if they choose.
We have as a goal to become a community resource with community
contributions from a variety of instructors and students.
This combination of use and contribution should promote broad
``buy-in'' by the DSA and FLA instructional communities.
Since the early days of Java, developing AVs has been used as training
for students seeking to learn web technologies, but unfortunately such
independent efforts often are soon lost~\cite{Shaffer10}.
The OpenDSA project provides a framework within which students across
the world could develop visualizations and contribute them to the
broader community.

OpenDSA disrupts the fundamental balance today between students,
textbooks, course content and activities, and instructors.
This approach of combining tutorial content with automated assessment
makes OpenDSA's approach a potential solution to a major problem with
another disruptor of education: the MOOC.
MOOCs often suffer from a lack of meaningful assessment and coursework.
OpenDSA provides a meaningful way to give students practice material
and instructors an assessment mechanism in a way that scales up to
MOOC-sized classes.
Our fundamental research questions relate to studying the effects of
this change in the balance, both on the part of students and on the
part of instructors.
How will OpenDSA impact student learning?
How can the OpenDSA environment be used by instructors to support
learning?
How will eTextbooks encourage instructors to make changes in instruction?
How can an eTextbook be disseminated?

\section{Prior Work}
\label{PriorWork}

Programmed Learning, Programmed Instruction, Keller plan, and Integrated
Learning System are concepts that formed the foundation for Computer
Based Training (CBT) in the early to mid 1980s~\cite{Lockee2004}.
The concept was originally based on B.F. Skinner's work~\cite{Skinner}.
These ideas are still used, for example in the
National Education Training Group (NETg) series from
Thomson/Course Technologies.
Typically such implementations are used by industry (HR departments)
for employee training, and do not involve a human instructor.

Most effort by CS educators involved with interactive AVs has focused
on developing AV technologies and the AVs themselves.
At best there have been small, focused tutorial modules that
incorporate AVs into a particular topic or small set of topics
within a course.
For example, \cite{Ross02Theory} reports on
web pages integrated with applets that were used in parts of a theory
of computation course.
Many instructors who teach FLA now use Rodger's JFLAP
(as reported in \cite{CSK11}, JFLAP is one of the most widely used
tools for FLA).
In the preface of Rodger's user manual on JFLAP~\cite{RoF06}, 
Rodger warns the reader ``our book assumes
that the reader has read briefly about these topics first in an
automata theory textbook or a compiler textbook''.

Ross~\cite{Ross08} describes an infrastructure that uses Perl and
Dreamweaver to create hypertextbooks.
Parts of a theory of computing text
and a biology book were produced using this technology.
Unfortunately, the system requirements for these hypertextbooks
hinder their wide adoption due to browser restrictions and use of Java
applets.
R{\"o}{\ss}ling and Vellaramkalayil~\cite{Roessling:Moodle} report on
integrating AVs into Moodle-based lessons,
but the emphasis is again on a technology that would support
visualization-based hypertextbooks in Moodle.
To our knowledge little progress has been made on the
actual writing of such a textbook.

Titterton, Lewis, and Clancy \cite{titterton2010experiences} have
used a lab-centric mode of instruction for introductory CS courses at
UC-Berkeley.
Their current work is being done in Moodle and uses a
small amount of AVs along with many ``check point'' exercises
that students must complete as they progress through the material.
It is built upon an earlier technology called UC-WISE that was
developed and used at UC-Berkeley.
Their materials are designed for introductory CS courses that aimed
mostly at developing students' programming skills.
Hence they make only limited use of AVs.
Alharbi, Henskens, and Hannaford~\cite{Alharbi2010} are using eXe,
an open-source authoring system that allows instructors to create
academic-related web content using XML and HTML
(\url{http://exelearning.org/}).
Their efforts intend to help teach an Operating Systems course using
visualization.
Learners can interact with the AVs, and can take quizzes and tests
online.
They used the Sharable Content Object Reference Model
(SCORM: \url{http://www.adlnet.gov/}) to support integration of
digital teaching materials with a CMS (Moodle in their case).

Karavirta~\cite{Karavirta2009} began to integrate AVs with hypertext
using HTML and JavaScript, allowing the hypertextbook to be viewed
in any browser without additional plug ins.
Learners can interact with the animation and draw annotations on it,
but his system did not store the annotation, nor support quizzes and
tests.
Note that Karavirta is now working with OpenDSA.
He developed the JSAV support library, described in
Section~\ref{Preliminary}.
  
Miller and Ranum's interactive eTextbook for Python
programming~\cite{miller2012}, and
\emph{CS Circles}~\cite{pritchard2013}
by Pritchard and Vasiga are Python courses for novice programmers.
To the best of our knowledge, these are the closest projects to our
idea of an interactive eBook, and our interactions with their project
motivated our use of reStructuredText (reST)~\cite{RST} and Sphinx
(sphinx.pocoo.org) as our authoring system.
Miller and Ranum produced a complete book that includes
embedded video clips, active code blocks that can be edited within the
book's browser page by the learner, and a code visualizer
that allows a student to step forward and backward through example
code while observing the state of program variables.
Miller and Ranum's book runs on the Google App Engine and includes
assessment activities requiring students to write a small function to
perform a single action.
Their grading system is rudimentary and only provides students
with simple pass/fail feedback upon completing an exercise.
In our opinion, the most important thing missing from this effort is a
wider variety of automated assessment with immediate feedback.

\emph{CS Circles} is an exercise-centric
eTextbook for learning Python.
It uses the WordPress Content Management System for authoring
and the CodeMirror plugin~\cite{haverbeke2011} to allow in-browser
code editing and automated programming assessment.
Both Miller and Ranum's book and \emph{CS Circles} use Guo's online
Python tutor~\cite{guo2013}, an embeddable web program visualization
for Python.
The online Python tutor takes a python source code as input and
outputs an \textit{execution trace} of the program.
The trace is encoded in JSON format and sent to the user's browser for
visualization via HTTP GET requests.
The backend can work on any webserver with CGI support or on the
Google App Engine.

Large scale use of educational hypermedia in South Korea
provides interesting feedback about the challenges of eTextbooks.
Kim and Jung~\cite{Korea} identify usability, portability,
interactivity, and feedback as major elements to consider while
designing such systems.
Learners should be able to ask questions and receive help, as well as
control, manipulate, search, and browse the eTextbook content.
They also advocate for the development of models to support group
collaboration.

To our knowledge, OpenDSA is the first effort to write a complete
eTextbook tightly integrated with AVs and automatically
assessed exercises that could be used as the primary learning resource
in a semester-long computer science course.
This is surprising because Marc Brown's groundbreaking
dissertation on AV from 1988~\cite{Brow:1988} states:
``Much of the success of the BALSA system at Brown [at the time] is
due to the tight integration of its development with the development
of a textbook and curriculum for a particular course.
BALSA was more than a resource for that course --
the course was rendered in software in the BALSA system.''
Some projects have developed AVs for all the topics of a course, such
as JFLAP, which covers topics for a full course on FLA,
and JHAV\'{E} and others for DSA topics.
But none of these replace or otherwise integrate with tutorial
information (i.e., the textbook).
The closest that we are aware of is the AlVIE project~\cite{Alvie},
which has AVs referenced directly from the (paper) textbook.

Why have AV developers not heeded Brown and
authored complete eTextbooks with integrated AVs and exercises?
Why have they instead focused on developing AV technologies, separate
from the tutorial content? 
The answer is that creating an active eTextbook is a huge amount of work.
DSA and FLA content is especially difficult to develop.
Not only have CS educators as yet been unable to  
author such an eTextbook, but commercial publishers (such as Zyante)
that have successfully used interactive technologies for lower-level
courses are finding the needs of a DSA course are much more complex.
Another impediment had been technical.
Java, JavaScript, and Flash collectively were a big step forward in
providing cross-platform interactivity.
HTML5 provides a further lowering of technical hurdles, making the
development task less daunting by permitting one implementation to run
on all major browsers, tablets, and even many mobiles.
The vision for what an eTextbook can bring to Computer Science courses
is laid out in a recent ITiCSE Working Group report~\cite{KorhonenWG13}.

Sophisticated automated exercise environments have been
developed to support K12 math instruction,
such as IXL~\cite{IXL} and Khan Academy~\cite{KhanAcademy}.
Many know Khan Academy (KA) for its collection of video
lectures on a wide variety of topics.
Equally important is the rich infrastructure and large collection of
exercises that allow students to practice math problems.
The KA infrastructure is especially notable since it is open source,
with the exercises themselves implemented in HTML/JavaScript.
Simple exercises include multiple choice,
fill-in-the-blank, and ordering a set of choices.
But the software infrastructure also allows developers to program
unique interfaces and activities for specific exercises.
The main limitation is only the designer's imagination for how to
create an exercise such that the answer can be automatically graded.
These capabilities are important to OpenDSA, since they allow us to
use the KA exercise infrastructure to create exercises that
are relevant to DSA and FLA rather than K12 math.
The KA organization has attracted a significant
number of volunteers who develop and contribute new exercises.
We believe that both the exercise infrastructure and the volunteer
contributor model are appropriate for developing OpenDSA.

\section{Preliminary Progress on OpenDSA}
\label{Preliminary}

Project PIs Shaffer, Naps, and Rodger bring unique resources and
experience that make this project feasible.
We all have extensive experience both with creating relevant
content~\cite{ShafferText,ShafferTextThird}
and with AV courseware, including the
JHAV\'{E}~\cite{naps1992pascal,naps1992pseudocode,pothering1995cpp,Naps05}
and JFLAP~\cite{JFLAPsite,RoF06}
systems, and the AlgoViz Portal~\cite{AlgoViz}.
Naps has coordinated two ITiCSE Working Groups on
AV~\cite{NapsPanelShort,NapsShortImpact} and two
on eTextbooks~\cite{Roessling:VizCoSH,KorhonenWG13}.
Between us we have active collaborations or extensive
interactions with most of the major AV developers in the world.
We have had some success in leveraging this extensive network of
developers to help with OpenDSA (most importantly with the Aalto
University group, but the author list of~\cite{KorhonenWG13} gives a
good indication of our current interactions).

Since 2011, Shaffer has worked with Ville Karavirta of
Aalto University on JSAV, a JavaScript-based AV library
that is meant to be the foundation for building AVs for the active
eTextbook~\cite{JSAVsite,Karavirta:ITiCSE13}.
Dr. Karavirta was a major developer for the TRAKLA2
system~\cite{Malmi04,TRAKLAurl}, and TRAKLA2 serves as a key
inspiration for our vision with its concept of
``proficiency exercises''.
Under funding from NSF TUES and NSF EAGER, we have already developed a
substantial amount of content including class-tested chapters on
linear structures, binary trees, sorting, and
hashing~\cite{OpenDSAsite}.
We encourage reviewers to look at a sample book instance available at
the OpenDSA website, \url{http://algoviz.org/OpenDSA/Books/OpenDSA}.

To the reader the fundamental unit of OpenDSA is the module, which is
a single browser page containing text, graphics, visualizations, code
snippets, and exercises of various types.
A module corresponds to a section in a traditional
textbook or a topic that might be covered in a single lecture or part
of a lecture.
They are generally equivalent to around 15-30 minutes of
instruction, though some modules are much shorter.
Modules can be grouped to form ``chapters''.
Modules are authored using reStructuredText~\cite{RST} (reST).
ReST is a so-called ``lightweight'' markup language, originally
designed to produce Python program documentation.
Sphinx is the engine that converts reST files to HTML, LaTeX, EPUB,
or PDF documents according to specific ``directives''.
A directive in Sphinx is equivalent to a macro in Microsoft Office
documents or a markup command in LaTeX.
A directive is implemented by a small Python program that controls how
Sphinx should process the text included in the directive.
We wrote several directives that augment support for creating
book-length documents that Sphinx currently lacks.

OpenDSA includes a configuration mechanism that allows instructors to
select from among existing modules to compile into an
eTextbook ``instance'' that contains only those modules that they wish
to use for their course.
Based on the configuration file, the module source and the indicated
collection of independent visualizations and exercises are compiled
together to produce a set of HTML pages (one per module) that define
an instance of an OpenDSA eTextbook.
Navigation through a given eTextbook is currently done through an
index of modules that looks much like a traditional table of contents.

OpenDSA uses a client-server architecture.
A content server delivers the HTML documents along with embedded
visualizations and exercises that make up the eTextbook.
The data collection server is a web application that we built using
the Django web framework with a MySQL database for data persistence.
We designed an API to support communication between the OpenDSA
``front end'' (the content that runs in the student's browser),
and the data collection ``back end''.
The back end collects the necessary information regarding student
progress to provide a history of completed exercises for use by the
student and the instructor.
We also collect fine-grained user interaction data.
Such log data can help debug usability or pedagogical problems with
the tutorials, or guide redesign to discourage pedagogically poor
student behavior~\cite{Breakiron2013}.
Additional information on the OpenDSA architecture can be found
in~\cite{Fouh14}.

During Fall 2012, Spring 2013, and Fall 2013, OpenDSA materials have
been used in classes at five universities in three countries,
and OpenDSA is being used in three additional universities during
Spring 2014.
We have conducted a number of initial evaluations of OpenDSA, both for
pedagogical effectiveness and for user satisfaction by students and
instructors~\cite{Hall13}.
OpenDSA was used to replace three weeks worth of standard lecture
materials on sorting and hashing during October 2012 by around 60
students in a DSA course at Virginia Tech.
Students in the treatment section (that used OpenDSA) scored about one
half of a standard deviation higher on the resulting exam than the
control group (that did not use OpenDSA), but this was not
statistically significant.
Student evaluations on OpenDSA from every class over each semester of
use have been highly positive.
During Fall 2012, mean scores on preference for interactive online
tutorials as compared to standard lecture actually went up after
the students had experience with OpenDSA materials.
Instructors have found that OpenDSA allows them to spend a greater
fraction of lecture time on content related to the more abstract and
difficult topics, and less on the mechanics of the algorithms.
More details on our Fall 2012 evaluation can be found in~\cite{Hall13}.

\section{JFLAP}
\label{JFlap}

JFLAP began in 1990 as a tool for pushdown automata written in C++ and
X Windows, then added finite automata and Turing machines to become
FLAP.
Stand-alone tools for topics such as construction proofs, grammars,
parsing algorithms, L-systems, and other models of Turing machines
were eventually merged in.  
In 1996 we initiated a Java version of FLAP, called JFLAP.
In 2002, a new interface and new material was added such as
Moore and Mealy machines, CYK parsing, and graph layout algorithms for
automata.
In 2011, a redesign of JFLAP was started to allow for a more flexible
alphabet and several new features that should be completed in 2014.
In addition, Rodger is writing a static FLA textbook that integrates
JFLAP into the presentation.
Students read the textbook and try JFLAP exercises by loading JFLAP
files or creating new examples. 

JFLAP is used around the world in FLA courses.
The \url{jflap.org} site has over 460,000 visits since 2005, and by
2006 JFLAP had been downloaded from over 160 countries.
JFLAP was one of two finalist candidates in the NEEDS
Premier Award for Excellence in Engineering Education Courseware
competition in 2007.
The submission packet contained letters from over 60
faculty in support of JFLAP.
There are nine books that use JFLAP in some way
including~\cite{Lin11,Moz10,Gop06,Ben06,God08}.
There are over fifteen JFLAP papers written by Rodger such
as~\cite{RQS11,RLR07},
and over twenty-five papers written by others that use JFLAP in some
way such as modifying it for blind students~\cite{CRA12} or using it
in a class~\cite{Nef10}.
In 2005-2007 Rodger conducted a study on JFLAP with fourteen
universities. 
The majority of students indicated that having access to JFLAP
made learning course concepts easier, made them feel more engaged in
the course, and made the course more enjoyable ~\cite{RWL09}.

With the software and the learning materials separate, students have
to run the software and read along in the book, or switch 
between an online textual copy and the software.
JFLAP is written in Java and does not run on mobile devices or tablets.
The logical next step is to fully integrate text with visualizations
using HTML5.
OpenDSA will provide the infrastructure for completing such an
eTextbook.
The current static learning materials rewritten using
OpenDSA authoring support and automated assessment will be built into
JFLAP practice exercises with OpenDSA infrastructure. 

We propose a three-year plan for converting the main parts of JFLAP
into an eTextbook using the OpenDSA structure.
The first year will focus on types of automata, the 
second will focus on proofs with automata, and the third year will
focus on  grammars, parsing and associated  proofs.
In parallel with the HTML5 implementation of JFLAP, each year the
learning materials and assessments for the topics will be
integrated into OpenDSA.

\section{Research Agenda}
\label{sec:plan}

With existing support from NSF and volunteer efforts from a number of
collaborators, OpenDSA is off to a good start in building the
base infrastructure and initial content.
We are somewhat ahead of the schedule laid out in the
proposal that lead to our original TUES award.
However, much remains to be done with content, infrastructure, and
pedagogical studies, and many new opportunities have arisen as
detailed below.
The main research objective of the proposed project is to test the
efficacy of the integrated OpenDSA courseware and determine the
circumstances under which success can be achieved.
In this section we describe the specific efforts to be made over the
three-year life of this award.
We present this in three broad areas of content development,
infrastructure development, and pedagogical studies.

Our goals include integrating certain existing AV-related projects
into OpenDSA: JFLAP and TRAKLA2 (both described above) and
JHAV\'{E}POP (described below).
Doing this goes beyond just improving the original projects or giving
them more exposure through reimplementation in HTML5.
They have been carefully selected because they provide important
activities for students within the context of DSA and FLA courses.
Integration of these activities within OpenDSA creates a learning
environment that is more than the sum of its parts.

Throughout the development of OpenDSA, we have been guided by the
principles of cognitive theory of multimedia presentation espoused
by Richard Mayer~\cite{Mayer02,Mayer08}.
Mayer has synthesized from pedagogical literature a set of principles
that can inform the design of instructional multimedia such as
OpenDSA.
Derived from Cognitive Load Theory~\cite{Sweller1999}, it aims to
maximize the benefits of engaging both the learner's verbal and visual
channels, while minimizing distracting and extraneous material.
The most compelling aspects of the principles are
(1) they provide practical advice that can be used by designers, and
(2) they are backed up (collectively) by literally hundreds
of individual studies that measure learner gains.
Mayer's work directly influences some of the goals listed next.

\subsection {Content and Presentation}

We have gone through three stages in our philosophy of presentation.
When we first began the OpenDSA project, our primary goal was to
tightly integrate textbook content with AVs, in order to solve
problems with student comprehension of the dynamic processes
associated with DSA.
We soon realized that making students engage the material through a
series of interactive exercises that provide immediate assessment and
feedback is even more important than the visualizations.
The third stage in our evolution came when we realized that visual
presentation of all aspects of the content tends to be more effective
than textual presentation.
This means that, to the greatest extent possible, the textual content
should be shifted into visuals supported by small amounts of text.
This is most typically expressed in OpenDSA as a series of
``slideshows'' embedded within the modules, whose pacing is controlled
by the student.
This more visual approach aligns well with Mayer's principles.
Our evolution can be seen by comparing the current presentations on
Sorting (the first OpenDSA chapter that we created) with the chapter
on Linear Structures (one of the more recent chapters), which makes
far more use of a series of small slideshows.

We believe that we have a good understanding of how to use (algorithm)
visualization to present the behavior of dynamic processes as embodied
by algorithms and data structures.
What we have so far failed to achieve is a better approach to
presenting conceptual material, especially the large body of
analytical reasoning that can be the most difficult aspects of DSA
or FLA courses for students to grasp.
In particular, students have trouble understanding the fundamental
principles of algorithm analysis theory, or its practical application
to analyzing particular algorithms.
We believe that we can make good progress by studying various
instances of visual proofs that exist in the relevant math and
computing
literature~\cite{Goodrich1998,Thompson2011,Blaheta2009,Sher2008,Hammack2006},
and extracting a collection of principles that we can use to create
effective visualizations of analytical material.
We seek to develop a rich collection of techniques and supported
graphical primitives to augment our presentations.
We seek to address some especially hard topics in a
standard DSA course.
Of special interest to us are good tutorial presentations for
recursion, algorithm analysis, and NP-completeness.
In particular, for recursion we have a vision for an interactive
tutorial driven by practicing many small programming exercises, under
the guidance of an adaptive tutoring system that takes students
through a progression of increasingly difficult material as they
become ready for it.

Our most pressing goal in terms of content is to complete a full
semester course of OpenDSA materials where the content is delivered
using as much visual support as possible.
This means re-writing some of our earlier efforts (such as the Sorting
chapter), creating new content, and further developing our repertoire
of visual presentation techniques and tools.

Mayer's work provides much support for
the incorporation of audio narration to replace
(note, \textbf{not} duplicate!) textual narration of the content
associated with the visuals.
This would naturally fit with the current slideshow-based presentation
where a typical slide presents a (text) sentence with a visual such as
the current state of a data structure.
Thus, our first experimentation will develop audio narration for
aspects of the tutorials.

\subsection {Infrastructure Development}
\label{plan:infrastructure}

OpenDSA content is comprised of a large collection of modules, where a
specific module typically has others as a prerequisite.
Collectively, the modules and their prerequisite structure defines a
directed graph that could be viewed as a concept map.
This approach is inspired in part by the Khan Academy (KA) Knowledge
Map (\url{http://www.khanacademy.org/exercisedashboard?k}), which
shows a directed graph with the prerequisite relationships for
a large body of exercises for K12 mathematics.
In our vision, each node on the DSA concept map corresponds to a
tutorial where content is integrated with a collection of assessment
activities for a particular topic.
A typical semester course might include 50-100 modules.
We have begun preliminary work toward this goal as part of our
EAGER/SAVI award, in conjunction with Sadhana Puntambekar of
University of Wisconsin-Madison.
Another use for a concept map is to reorganize the glossary,
which is currently an alphabetical listing of terms with definitions.
Organizing the glossary as a concept map would make it easy for
students to explore the connections between related terms.

Given a rich collection of modules and definitions for their
prerequisite relationships, an instructor should be able to use 
a simple interface to select a subset of modules to make up a given
course.
The selected modules can be processed to generate a ``book instance''.
We currently support this ability through an XML-based configuration
file.
The next step is to provide a GUI for easily creating the
configuration file.

Instructors that have used OpenDSA have requested a mechanism for
creating coursenotes for use in class presentations.
While the AVs make for good in-class presentation tools, the module
format does not directly lend itself to classroom presentation.
A mechanism for authoring course slides (in reST) will be created that
builds on the existing Sphinx slideshow theme.

We currently generate a massive amount of interaction log data.
We need improved tools to turn low-level (syntactic) interactions into
semantic meaning.
For example, how much time do students spend on
various exercises or slideshow presentations?
Once we have improved log analysis tools, we seek to
take what we learn from the log data to encourage best
learning behaviors.
Prior experience shows that we must be careful to prevent students from
abusing the ability to repeat exercises.
If it is possible to achieve completion of a component by guessing,
even if it is likely to take additional time, then
some students will take that approach~\cite{Karavirta2005}.
We also need to make sure that students do not find the
exercises to be tedious or unreasonably time consuming.
There are many fine details to work out.
For example, the proficiency exercises typically require students to
successfully perform a series of steps correctly.
Naively, grading might simply count the number of steps done
correctly.
Unfortunately, if the student makes a mistake at any point in the
process, it might be impossible to recover and get any additional
steps correct, leading to a non-representative assessment of their
performance.
One solution is to always bring the student back ``on track'' when an
incorrect step is made.
But if this is not done carefully, it might lead students to guess.

Our approach to key topics like recursion requires that we
deliver and automatically assess small programming exercises.
Standard test-case driven evaluation is already supported by OpenDSA
in rudimentary form.
We envision a more visual approach, as implemented in the
JHAV\'{E}POP~\cite{JHAVEPOP} exercise system developed at University of
Wisconsin--Oshkosh.
There, the student's solution is presented back to the student
in the form of a visualization generated from the student's code.
In the original JHAV\'{E}POP, the student must decide if their answer
is correct.
We seek to tie this approach in with automated assessment of the
exercises.
This will be particularly useful for teaching topics like pointer
manipulation.

\subsection{Pedagogical Studies}

OpenDSA provides the opportunity to change pedagogy away from
lecture-and-textbook.
Making the course more interactive requires
mapping the current lecture course into an interactive
format to ensure alignment across the associated course objectives.
Students and instructors face many challenges when making a
transition from lecture-based courses to any format where students are
expected to spend significant time on their own working through
materials and tutorials.
In studies involving similar effort~\cite{Henry2010},
the faculty report some pushback from students,
because in a face-to-face course instructors typically do more of
the delivery and interaction with the content.
We expect that when a course is organized around an active eTextbook,
students will spend more time out of class with the material.
We want them to become involved in the learning in a
deeper and more active way than they were in the past.
Success is related to students' level of self-directed learning and
self-regulated learning (SRL)~\cite{Muis2007,Hofer1997}.
SRL affects how they manage the work required by the new format, how
they interpret the new assessment mechanism,
and their understanding of the shift in the role of instructor to
facilitator.

Henry, et al.~\cite{Henry2010} showed that students accustomed to
traditional instructional strategies often require mentoring about how
to direct their own learning of the material and perform the work.
How do we build student's ability to self regulate? 
Self-directed learning represents a paradigm shift for
many students since the study habits that have brought them success in
traditional learning environments are not always effective in the new
settings~\cite{Hmelo-Silver2004}.
Our evaluation of the process therefore has to be formative and
promote continued improvement in student performance.
We will design effective feedback rubrics and assessment methodologies
that are replicable and scalable.

We can hypothesize that instructors will start by using the eTextbook
as a direct replacement for text and paper homework, and gradually
evolve into larger changes in how they conduct their class.
While OpenDSA enables pedagogical changes, does its availability
encourage instructors to greater pedagogical experimentation?
Does it encourage better teaching practices?
Does it lead to more interactive classes?
We will study and characterize the evolution of instructors' changing
pedagogy with eTextbooks.

\section{Implementation and Impact Study}
\label{EvalSec}

Specific research questions and metrics for the implementation and
impact study are discussed in Table~\ref{Exhibit}.
OpenDSA courseware will be evaluated in DSA and FLA courses at  
Virginia Tech (CS3114), Duke (Compsci334), and UW-Oshkosh (CS271),
as well as at a number of other institutions.
Due to its large enrollment, CS3114 is given to two sections each
semester.
Similar to experiments in Fall 2012, these class sections can be used
to group students into control and experimental cohorts, with control
groups receiving traditional, lecture-based instruction while the
experimental group uses OpenDSA courseware.
The large enrollment provides an opportunity to simultaneously compare
new and old strategies while minimizing confounding factors.

From a teaching perspective, OpenDSA tutorials deliver course content
incrementally, and balance teaching the content processes with
interactive, auto-assessed activities that provide immediate feedback.
From a learning perspective, they are intended to move students from a
passive stance in a lecture-type classroom setting to an active
position of constructing learning and tracking their own comprehension
through immediate feedback received from the exercises.
The theory of change driving the design and implementation of these
tutorials is to encourage students' engagement with the content
and involve them in the assessment loop as active participants in such
a way that they, as well as their instructors, know that they are
learning.
According to Bandura's cognitive theory of
self-efficacy~\cite{Bandura},
instruction that allows students to check their own progress at a
designated level of proficiency positively impacts outcome.
Constructivist theory suggests that timely feedback
can encourage students to modify their work.
Lovett and Greenhouse~\cite{Lovett} show that receiving feedback on
the process of learning gives significant improvement as compared to
only receiving feedback from the instructor on performance outcomes.

The following research questions will be explored through a study
sample of courses taught by instructors recruited at the SIGCSE annual
meeting, through the SIGSE listserv, and by personal contacts to
project participants.
Additionally, a control sample will be selected and matched
within institution where number of course sections permit,
or alternatively based on an array of population, geographic, and
institutional mission indicators identified though features of the
study sample.
As part of our evaluation, we have requested funding to support
instructors from outside the project who will adopt OpenDSA materials
for their courses, and collect implementation and impact data for us.
We have budgeted \$10,000 each year as stipends to support two outside
adopters, for a total of six over the three years.
While far more instructors than this are expected to use OpenDSA
during this time,
these instructors will be supported to both tailor OpenDSA content
to their classes, and to conduct assessment beyond that
normally expected from instructors who use OpenDSA.

\begin{table}
\caption{Research questions and data sources.
Pre: Pre-Assessment test.
Post: Post-Assessment test.
AUD: Analytic User Data.
SI: Student Interviews.
MI: Midterm Interviews.
EI: End-of-Semester Interviews.
TL: Teacher Log.
}
\label{Exhibit}
\begin{tabular}{|l|c|c|c|c|c|c|c|}
\hline
Data Source&Pre&Post&AUD&SI&MI&EI&TL\\
\hline
\textbf{RQ1} Does OpenDSA increase student proficiencies&&&&&&&\\
with core data structures and algorithms content?&X&X&&&&&\\
\hline
\textbf{RQ2} Does OpenDSA student proficiency exceed&&&&&&&\\
that of students not exposed to OpenDSA?&&&&&&&\\
\hline
\textbf{RQ3} How do students use OpenDSA materials?&&&X&&&&\\
\hline
\textbf{RQ4} What factors led to instructor selection and&&&&&&&\\
implementation of OpenDSA materials?&&&&X&&&\\
\hline
\textbf{RQ5} Do instructors change their use of OpenDSA&&&&&&&\\
materials over time?&&&&&X&X&X\\
\hline
\textbf{RQ6} Are DSA fundamental course structures&&&&&&&\\
altered based on implementation of OpenDSA?&&&&&&&X\\
\hline
\textbf{RQ7} Does OpenDSA use and/or course success vary&&&&&&&\\
across student demographic and social dimensions?&X&X&&&&&\\
\hline
\end{tabular}
\vspace{-\bigskipamount}
\end{table}

Data sources are tagged to key research questions in the table.
\textbf{RQ1} and \textbf{RQ2}:
A parallel item cognitive pre-assessment and post-assessment will be
administered to the study group and the comparison group to measure
degrees of student proficiency concerning DSA and FLA content.
OpenDSA materials will consist of an open test bank and a
closed test bank.
The open test bank can be drawn from OpenDSA exercises, but will
also be provided to the control group to use for review and
assessment preparation purposes.
The closed bank will be electronically administered with a
proctor and used to measure progressions in core knowledge.
\textbf{RQ3}: Analytic user data will consist of access frequency,
overall time on task, and individual lesson completion time.
These data permit the research team to gauge student use
of OpenDSA as well as identify specific content that
students collectively find difficult to understand.
In addition to user data, randomized study sample site
participants will be selected for small group interviews concerning
the nature of OpenDSA materials use.
\textbf{RQ4}, \textbf{RQ5}, and \textbf{RQ6}:
Instructor implementation information will be gathered through a
biweekly teacher log, a total of six entries per instructor, as well
as midterm and end-of-semester follow-up interviews.
Information such as content covered, course format, course learning
environment, course activity, and course pace will be collected.
\textbf{RQ7}: Data sources from \textbf{RQ1} and \textbf{RQ2}
will be used to determine if OpenDSA materials have differential
impacts on student academic outcomes and tendencies of integrated
courseware use for individuals with different demographic
characteristics such as females, underrepresented ethnic minorities,
and first generation college students.

Each data source will be collected in each project year.
Study sites implementing OpenDSA will contain two student study
samples (Spring and Fall semesters) each project year.
Pre/post assessment data, analytic user data, small group student
interviews, midterm and end-of semester interviews, and teacher logs
will be collected or conducted from/for all study sites.
Similarly, the control sites will also contain two student study
samples each project year where pre/post assessment data and teacher
log data will be collected for non-treatment comparison.

The evaluation plan requires systematic data collection at multiple
sites, along with planning, merging, analyzing, and documenting the
data collected in these sites.
We will conduct validity and reliability tests and report the
psychometric quality of the instruments developed because aspects of
this project require development of new tools and assessment.
The plan is sufficiently complex that we have requested support for an
external project consultant to execute the plan.

Project evaluation provides continuous, timely, and constructive
feedback through multiple evaluations during the course of the
project, suggesting program changes, if necessary, to ensure
project success.
The evaluation plan is guided by these goals:
\begin{enumerate}
\itemsep=0pt
\parskip=0pt
\parsep=0pt
\vspace{-\medskipamount}

\item Determine whether project goals and benchmarks were
reached concerning materials development, research, and
dissemination.
\item Determine whether the specified research methods were employed
  during the implementation and impact study.
\item Obtain feedback from students and teachers on their
perception of the overall effectiveness of the OpenDSA materials.
Interviews will be conducted with selected teachers and students.
\item Measure the degree of transfer of the OpenDSA materials.
\end{enumerate}

\section{Dissemination Plan}
\label{DissemSec}

The ``open'' aspect of OpenDSA means that instructors themselves
can become active participants in designing and contributing to the
content that OpenDSA offers to their students.
This involvement can come in many ways:

\begin{itemize}
\itemsep=0pt
\parskip=0pt
\parsep=0pt
\vspace{-\bigskipamount}

\item Given that OpenDSA is not a linear sequence of material but
rather a ``bag of instructional resources'', instructors can
define their own specific paths through the instructional modules,
encapsulated as eTextbook ``instances''.
These instances can then be distributed as OpenDSA configuration files
for others to use.

\item Instructors are able to contribute their
own content to OpenDSA.
These contributions might be any artifact type supported by OpenDSA
such as text, visualizations, interactive activities, or exercises.
  
\item The contributions might be original (at least to OpenDSA),
or modified from materials already existing in OpenDSA.
\end{itemize}

\vspace{-\bigskipamount}
To encourage faculty to become involved in designing and creating
OpenDSA content, we will offer workshops at conferences such as
SIGCSE.
Such workshops expose participants to materials that we have already
developed, which aids adoption.
They also show participants how to design and contribute their own
materials.
We anticipate doing workshops during each year of the project.

In addition to the workshops, we will also hold ``advisory panel''
meetings at SIGCSE that will be analogous to the stakeholder meetings
for our AlgoViz Portal project that we hosted in 2008 and 2009.
Such meetings involve individuals who have already become active in
the OpenDSA community.
The AlgoViz stakeholder meetings turned out to be extremely important
in the evolution of that project and in its subsequent success.
For OpenDSA, involving key players in the AV development community is
an important part of generating a robust community of content
developers.
We therefore include in our budget support for both the workshops
and the stakeholder meetings at SIGCSE during the years of the
proposal.
We include a request for some international travel support
since some of our most important collaborators are located in Europe
(see for example the support letter from Aalto University).

One aspect of dissemination is the support that instructors need when
using online materials with automated assessment.
They need support for collecting and maintaining the (auto)
graded material, and the ability to track progress for their
students.
OpenDSA currently provides this support as part of its server-side
infrastructure.
However, while it is feasible for an academic project to
develop high-quality instructional material, and feasible
for an academic project to make such materials available for a period
of many years, it is not likely to be feasible for an academic project
to sustain the infrastructure needed to support a large body of
students (potentially 10s or 100s of thousands per year) and their
instructors.
This is the sort of service that is more appropriate for a commercial
entity, providing hosting and back-end data collection as a service.

Our vision requires that OpenDSA content always be open, and
to be viewed as a community-based resource.
However, we also seek to work with commercial service providers on a
``value added'' basis, where the provider hosts OpenDSA book
instances, and maintains student progress and instructional
gradesheets in a FERPA-compliant manner.
To this end, we have begun discussions to partner with Zyante
(\url{http://www.zyante.com/}) to provide materials for their DSA
offerings.
We have a preliminary analysis of the infrastructure changes
that will be needed for OpenDSA materials to communicate with
Zyante's infrastructure.
While not trivial, the necessary APIs can be designed that
allow OpenDSA to both be usable in its open form, or
used through Zyante (or potentially other service providers) with the
instructor services added at a reasonable cost to students.
This model could prove decisive to the future of eTextbooks and
online courses in general.

\section{Project Management}

OpenDSA is a collaboration of several universities,
including the three directly supported under this proposal.
Dr. Shaffer will be responsible for coordinating the activities of the
various participants.
Dr. Ernst is experienced in educational research and development
initiatives and will be responsible for phases of the project
including data collection and reporting.
They will work with the other PIs to develop the detailed instruments
and collect the necessary data, as described in Section~\ref{EvalSec}.
We request funding for 2.5 years of Postdoc support.
The Postdoc will assist Dr. Shaffer in directing the software and
content development efforts, interacting with outside faculty who will
adopt and evaluate OpenDSA, and in conducting assessment activities
(under the supervision of Dr. Ernst).

Drs. Rodger and Naps will direct efforts at Duke and UW-Oshkosh,
respectively.
Dr. Rodger's will oversee reimplementation and deployment within the
OpenDSA framework of JFLAP.
She will perform evaluations at Duke with parts of the new JFLAP.
Dr. Naps will work with Dr. Shaffer to develop OpenDSA content,
consult on reimplementation of JHAV\'{E}POP,
perform evaluations at UW-Oshkosh, and coordinate with adopting
instructors.
We already routinely hold Skype conference calls to coordinate our
work, including with our partners at Aalto University.
We will formalize this process into a regular monthly Skype discussion
of the project PIs and other relevant participants.
The annual meetings to precede the SIGCSE conference will serve as
additional opportunities to interact in person with key stakeholders.

Key timing aspects are as follows.
Each year, we will recruit two outside faculty who will both
use OpenDSA in a course, and provide extensive evaluation feedback.
These individuals will be compensated by grant funds.
We expect that a broad group of faculty will also adopt OpenDSA in
their classes, but they will not have the same evaluation obligations.
Our postdoc will take on management responsibilities for project
development and implementing evaluation tasks.
The postdoc should be in place by the middle of the first year.
We will have our major stakeholder meetings at the annual SIGCSE
conference in early March of years two and three.
Some aspects of content development, infrastructure development,
and pedagogical studies will be ongoing in each of the three years.
But we expect to complete most of the remaining infrastructure
development as described in Section~\ref{plan:infrastructure} during
year one.
Year two will see completion of most of the content-related tasks,
with further polishing of materials in year three.
Evaluation and dissemination will be ongoing throughout the life of the
project, but will be the major focus of year three.

\section{Broader Impacts}

OpernDSA provides an interactive, visual presentation for dynamic
material (algorithms and operations on data structures) as well as
visual presentation of abstract concepts (such as algorithm analysis),
combined with immediate feedback through interactive exercises.
OpenDSA has the potential to improve the learning of students of DSA
and FLA courses in ways never before possible.
Beyond Computer Science, the models we provide for architecting,
using in class, disseminating, and assessing eTextbook materials are
broadly applicable across a range of STEM and non-STEM disciplines.

This project has had many undergraduate students work on AV-related
activities over the years.
PI Naps managed a site REU project from 2009-2011 that supported
5--10 undergraduates each summer to develop JHAV\'{E} modules.
PI Shaffer has had approximately 20 undergraduate independent study
students work on AV projects (9 during Spring 2014 semester alone),
and PI Rodger has supervised over 20 undergraduates on JFLAP
development.
We anticipate that many undergraduates will in the future choose to do
independent projects related to OpenDSA, including from other schools
not directly supported by the project.
Being an open-source project, OpenDSA provides a framework whereby
such students can productively contribute.
Shaffer has previously involved minority students through Virginia
Tech's Multicultural Academic Opportunities Program (MAOP) internship
program (equivalent to an NSF REU).
We have budgeted money each year at Virginia Tech to
support undergraduate researchers, including MAOP interns.
Duke and UW-Oshkosh budgets include funds for undergraduate research.


\section{Results from Prior NSF Support}

\textbf{NSF TUES Phase I Project (DUE-1139861)}
\emph{Integrating the eTextbook: Truly Interactive Textbooks for Computer
Science Education.} PIs: C.A. Shaffer, T. Simin Hall, T. Naps,
R. Baraniuk.
\$200,000, 07/2012 -- 06/2014.
\textbf{NSF SAVI/EAGER Award (IIS-1258571)}
\emph{Dynamic Digital Text: An Innovation in STEM Education},
PIs: S. Puntambekar (UW-Madison), N. Narayanan (Auburn),
and C.A. Shaffer (2013). 
\$247,933, 01/2013 -- 12/2014.
\textbf{NSF CCLI Phase 1 Award (DUE-0836940)}
\emph{Building a Community and Establishing Best Practices in
Algorithm Visualization through the AlgoViz Wiki.}
PIs: C.A. Shaffer, S.H. Edwards.
\$149,206, 01/2009 -- 12/2010.
\textbf{NSF NSDL Small Project (DUE-0937863)}
\emph{The AlgoViz Portal: Lowering Barriers for Entry into an Online
Educational Community.} PIs: C.A. Shaffer, S.H. Edwards, \$149,999,
01/2010 -- 12/2011.
\textbf{Intellectual Merit} 
The first two projects provided online infrastructure
(the AlgoViz Portal (\url{http://algoviz.org}) and related community
development efforts to promote use of AV in computer science courses.
This work was an important precursor to OpenDSA, as it
allowed us to interact with many CS instructors and AV developers,
leading us to an understanding of the fundamental missing parts in
existing DSA instruction.
They also initiated many of the international collaborations that lead
to OpenDSA.
Two journal papers~\cite{Shaffer10,Fouh:AV11} and five conference
papers~\cite{ShafferSIGCSE07,ShafferSIGCSE10,ShafferSIGCSE11,ShafferPVW11,ShafferKoli11}
have been produced relating to this work.
The second pair of (ongoing) awards support the initial phases of
OpenDSA, and an active collaboration involving Virginia Tech and Aalto
University (Helsinki), among others.
Publications related to this work so far
include~\cite{KorhonenWG13,Karavirta:ITiCSE13,Hall13,Fouh14}.
\textbf{Broader Impacts}
include dissemination of AV artifacts and DSA courseware to a broad
range of CS students, and made them available through the NSF NSDL.

\medskip

\textbf{NSF CCLI Phase 1 Awards (DUE-0341148 and DUE-0126494)}
\emph{Integrating Algorithm Visualization into Computer Science
Education.} PIs: T.L. Naps, S. Grissom, and M. McNally,
\$71,993 for 2001--2002 and \$197,118 for 2003--2007.
\textbf{REU grant (CNS-0851569)}
\emph{Exploring Open Source Software: Development and Efficacy of
Online Learning Environments in Computer Science},
PIs: T.L. Naps and D. Furcy,
\$261,167 for 2009--2012.
\textbf{Intellectual Merit} 
The first two grants resulted in significant development on the
JHAV\'{E} algorithm visualization system.
A website was established at \url{http://jhave.org} to disseminate
the visualization-based materials that were produced for teaching
data structures and algorithms.
The following papers were published as a result of the
grant, three had undergraduate students as
co-authors~\cite{furcy2007blocktree,lucas2003visualgraph,NaRoPVW:2006,GAIGS_support,Naps05,NaMcPVW:2006,GrMcNa:2003,NaGr:2002,RoNa:2002}.
\textbf{Broader Impacts}
The REU grant provided student participants with extensive exposure to
open-source software for the delivery of instructional algorithm
visualizations, as developers, authors, and educational researchers.

\medskip

\textbf{NSF ITEST DRL-1031351}
\emph{Collaborative Research: Scaling up an Innovative Approach for
Attracting Students to Computing} 
PIs: S.H. Rodger, S. Cooper (Stanford),
W. Dann (CMU), M. Schep (Columbia College), 
R. Stalvey (College of Charleston), P. Lawhead (U. Mississippi)). 
\$2,499,870 for 2011--2016.
This and award {\bf NSF ITEST 0624642} develop curriculum
materials for the Alice programming language, integrating computer
science into K-12.
\textbf{Intellectual Merit:}
We developed over 80 tutorials on CS and animation
topics for all K-12 disciplines~\cite{CDL11,RDG12},
teachers attending our workshops developed over 180 lesson plans
(see \url{www.cs.duke.edu/csed/alice/aliceInSchools}).
\textbf{Broader Impacts:} Summer workshops taught over 200 K-12
teachers programming and how to integrate computing into their
disciplines.
Published papers include 7 female undergraduates as co-authors, who
participated in teaching and mentoring the teachers.

\textbf{NSF TUES DUE-1044191}
\emph{Integrating Visualization and Interaction into the Formal
Languages and Automata Course}
PI: S.H. Rodger
\$199,996 for 4/2011-- 03/2015.
\textbf{Intellectual Merit:}
This and prior award {\bf NSF CCLI 0442513} focus on enhancing
JFLAP with new algorithms, designing curriculum materials, and running
studies on the usability of JFLAP in learning automata theory.
\textbf{Broader Impacts:}
JFLAP is used extensively around the world in automata theory courses.
The published papers~\cite{RWL09,RQS11} include five
undergraduate co-authors.

\medskip

\textbf{NSF EHR (DRL-1156629)}
\emph{Transforming Teaching through Implementing Inquiry project},
PIs: J. Ernst, L. Bottomley, A. Clark, V.W. DeLuca, S. Ferguson,
\$1,997,532, 2011--2015.
\textbf{NSF EHR (DRL-1135051)}
\emph{Re-designed High Schools for Transformed STEM Learning},
PIs: E. Glennie, J. Ernst, \$1,954,066, 2011--2015.
These projects showcase Co-PI Ernst's experience with developing and
assessing STEM education initiatives.
The first explores the use of cyberinfrastructure tools to improve
quality and enhance delivery of professional development materials for
grades 8-12 engineering, technology, and design educators.
The second is a longitudinal research project to assess
1) North Carolina New Schools Project (NCNSP) STEM strand school
student learning over time, using extant data, survey data, and
performance assessments, contrasted with student performance in a
matched set of comparison high schools using traditional curricula;
and
2) School-level policies and instructional practices that
schools employ to promote student learning through a series of case
studies involving site visits and teacher logs.
Publications include~\cite{Ernst12,Ernst13,Segedin13}.
\textbf{Broader Impacts:} These projects informed the NCNSP STEM
strand schools work with respect to implementation of the empirically
re-visioned STEM model. They have provided a cyber system and content,
focused on increasing engineering, technology, and design teacher
quality, that is readily adaptable to the full breadth of STEM
education implementation initiatives. 

%\newpage
%\input{managementplan}

\newpage
\pagestyle{empty}
\bibliographystyle{plain}
\bibliography{OpenDSA}

\end{document}
