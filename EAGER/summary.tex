% Proposal to be submitted to IUSE program
% http://www.nsf.gov/funding/pgm_summ.jsp?pims_id=504976
% Due February 4, 2014

\documentclass[11pt]{article}
\usepackage{fullpage}
\usepackage{graphicx}
\usepackage{url}
\usepackage{paralist}
\usepackage[normalem]{ulem}

\begin{document}

\newcommand{\NP}{\mbox{${\cal NP}$}}
\renewcommand{\floatpagefraction}{.90} % My preference
\renewcommand{\textfraction}{.10}

\pagestyle{empty}

\textbf{Project Summary} \\
\indent \textbf{Title:} TUES: EAGER: Scaffolding Big Data for Authentic Learning of Computing

\medskip

We propose to create curriculum and technologies that leverage ``Big
Data'' to create authentic and engaging learning experiences in
computational thinking and computer science courses.
We ask for support through the EAGER process because
we have a unique, time-sensitive opportunity at Virginia Tech to
profoundly impact the general education curriculum through a new
non-major's Computational Thinking course,
whose first offering will be in the Fall 2014 semester.
The materials developed for Computational Thinking
also will be used to improve introductory courses in Computer Science,
extending our prior work that used real-time data sources.
Further, collaborating with Virginia Tech's Department of Engineering Education,
we will explore how our work could support 
authentic and motivating introductory computing experiences for engineering students.

We address the critical challenge of weaving together curriculum,
pedagogy, and tools to engage groups of learners that have starkly
different interests and expectations about computation.
Our solution uses authentic project-based experiences that engage
multi-disciplinary groups of learners with computation.
We leverage the widespread availability of ``Big Data'' sources from the
Internet.
But to provide Big Data resources to students without a
programming background requires careful scaffolding of the technology
to manipulate Big Data streams.
With the proper scaffolding, meaningful and motivating projects can
easily be deployed by instructors to quickly and seamlessly incorporate
this compelling dimension into new learning experiences.
This proposal builds on the success of the RealTimeWeb project,
our framework for rapidly building real-time, web-data centered
assignments in introductory CS courses.

The proposed work will
(1) develop novel curriculum resources including interactive learning
materials integrating graphical programming, visualization, and
in-line execution;
(2) expand and enhance the RealTimeWeb with new Big Data streams and
interfaces to visualization and other critical services;
and (3) develop, apply, and analyze an extensive set of assessment
measures including assessments related to achievement of learning
objectives, student motivation, and the dynamics of student cohorts.

\textbf{Intellectual Merit:}
By introducing authentic, massively-sized datasets of relevance into
the early undergraduate curriculum, we can create a more engaging
experience for students where they develop a foundational knowledge of
Big Data concepts.
This project will provide an excellent opportunity to improve the
theoretical understanding of the problems and best practices for
teaching Big Data, even as it offers a practical method for individual
instructors to begin using these resources.
This project will also provide insights about the challenge of
designing engaging and relevant contexts for projects that positively
impact CS learning outcomes, while simultaneously enabling students to
work with one of the most important technologies of the modern era.

\textbf{Broader Impact:}
This project will improve recruitment, retention, and engagement of
students within STEM disciplines.
Prior research indicates that women are more likely to study and excel
in Computer Science when content is contextualized and proven useful
for solving real-world problems.
Non-major students can be given realistic assignments that can more
directly relate to their intended line of work, further increasing
their motivation to succeed during their time in a CS course.
Appropriately redesigning programming projects to involve interesting,
contextualized Big Datasets is likely to improve the relevance and
attractiveness of Computer Science to a wider community.

\end{document}
