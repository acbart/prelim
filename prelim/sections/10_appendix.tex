\appendix

\textbf{Appendix}

\section{Definitions of Computational Thinking}\label{app:ct-definitions}
    
    \subsection{AP CS Principles}
    
    \begin{description}
        \item[P1: Connecting computing] \hfill \\
        Developments in computing have far-reaching effects on society and have led to significant innovations. These 
        developments have implications for individuals, society, commercial markets, and innovation. Students in this 
        course study these effects and connections, and they learn to draw connections between different computing 
        concepts. Students are expected to: 
        \begin{itemize}
            \item Identify impacts of computing; 
            \item Describe connections between people and computing; and 
            \item Explain connections between computing concepts. 
        \end{itemize}
        
        \item[P2: Developing computational artifacts] \hfill \\
        Computing is a creative discipline in which the creation takes many forms, ranging from remixing digital music 
        and generating animations to developing websites, writing programs, and more. Students in this course engage 
        in the creative aspects of computing by designing and developing interesting computational artifacts, as well as 
        by applying computing techniques to creatively solve problems. Students are expected to: 
        \begin{itemize}
            \item Create an artifact with a practical, personal, or societal intent; 
            \item Select appropriate techniques to develop a computational artifact; and 
            \item Use appropriate algorithmic and information-management principles. 
        \end{itemize}
        
        \item[P3: Abstracting] \hfill \\
        Computational thinking requires understanding and applying abstraction at multiple levels ranging from privacy 
        in social networking applications, to logic gates and bits, to the human genome project, and more. Students in 
        this course use abstraction to develop models and simulations of natural and artificial phenomena, use them to 
        make predictions about the world, and analyze their efficacy and validity. Students are expected to: 
        \begin{itemize}
            \item Explain how data, information, or knowledge are represented for computational use; 
            \item Explain how abstractions are used in computation or modeling; 
            \item Identify abstractions; and 
            \item Describe modeling in a computational context. 
        \end{itemize}
        
        \item[P4: Analyzing problems and artifacts] \hfill \\
        The results and artifacts of computation, and the computational techniques and strategies that generate them, 
        can be understood both intrinsically for what they are as well as for what they produce. They can also be 
        analyzed and evaluated by applying aesthetic, mathematical, pragmatic, and other criteria. Students in this 
        course design and produce solutions, models, and artifacts, and they evaluate and analyze their own 
        computational work as well as the computational work that others have produced. Students are expected to: 
        \begin{itemize}
            \item Evaluate a proposed solution to a problem; 
            \item Locate and correct errors; 
            \item Explain how an artifact functions; and 
            \item Justify appropriateness and correctness. 
        \end{itemize}
        
        \item[P5: Communicating] \hfill \\
        Students in this course describe computation and the impact of technology and computation, explain and justify 
        the design and appropriateness of their computational choices, and analyze and describe both computational 
        artifacts and the results or behaviors of such artifacts. Communication includes written and oral descriptions 
        supported by graphs, visualizations, and computational analysis. Students are expected to: 
        \begin{itemize}
            \item Explain the meaning of a result in context; 
            \item Describe computation with accurate and precise language, notation, or visualizations; and 
            \item Summarize the purpose of a computational artifact. 
        \end{itemize}
        
        \item[P6: Collaborating] \hfill \\
        Innovation can occur when people work together or independently. People working collaboratively can often 
        achieve more than individuals working alone. Students in this course collaborate in a number of activities, 
        including investigation of questions using data sets and in the production of computational artifacts. Students 
        are expected to: 
        \begin{itemize}
            \item Collaborate with another student in solving a computational problem; 
            \item Collaborate with another student in producing an artifact; and 
            \item Collaborate at a large scale
        \end{itemize}
    \end{description}
    
    \subsection{CSTA}
    Computational thinking (CT) is a problem-solving process that includes (but is not limited to) 
    the following characteristics: 
    \begin{enumerate}
        \item Formulating problems in a way that enables us to use a computer and other tools to help solve them.
        \item Logically organizing and analyzing data
        \item Representing data through abstractions such as models and simulations
        \item Automating solutions through algorithmic thinking (a series of ordered steps)
        \item Identifying, analyzing, and implementing possible solutions with the goal of achieving the most efficient and effective combination of steps and resources 
        \item Generalizing and transferring this problem solving process to a wide variety of problems
    \end{enumerate}
    These skills are supported and enhanced by a number of dispositions or attitudes that are essential dimensions of CT. These dispositions or attitudes include:
    \begin{enumerate}
        \item Confidence in dealing with complexity
        \item Persistence in working with difficult problems
        \item Tolerance for ambiguity
        \item The ability to deal with open ended problems
        \item The ability to communicate and work with others to achieve a common goal or solution
    \end{enumerate}
    
    \subsection{Google Definition}
    Computational thinking (CT) involves a set of problem-solving skills and techniques that software engineers use to write programs that underlie the computer applications you use such as search, email, and maps. Here are specific techniques:\cite{google-computational-thinking}
    
    \begin{description}
        \item[Decomposition] The ability to break down a task into minute details so that we can clearly explain a process to another person or to a computer, or even to just write notes for ourselves. Decomposing a problem frequently leads to pattern recognition and generalization, and thus the ability to design an algorithm.
        \item[Pattern Recognition] The ability to notice similarities or common differences that will help us make predictions or lead us to shortcuts. Pattern recognition is frequently the basis for solving problems and designing algorithms.
        \item[Pattern Generalization and Abstraction] The ability to filter out information that is not necessary to solve a certain type of problem and generalize the information that is necessary. Pattern generalization and abstraction allows us to represent an idea or a process in general terms (e.g., variables) so that we can use it to solve other problems that are similar in nature.
        \item[Algorithm Design] The ability to develop a step-by-step strategy for solving a problem. Algorithm design is often based on the decomposition of a problem and the identification of patterns that help to solve the problem. In computer science as well as in mathematics, algorithms are often written abstractly, utilizing variables in place of specific numbers.
    \end{description}
    
    \subsection{Cuny, Snyder, Wing}
    Computational thinking for everyone means being able to 
    \begin{itemize}
    \item Understand what aspects of a problem are amenable to computation 
    \item Evaluate the match between computational tools and techniques and a problem 
    \item Understand the limitations and power of computational tools and techniques 
    \item Apply or adapt a computational tool or technique to a new use 
    \item Recognize an opportunity to use computation in a new way 
    \item Apply computational strategies such divide and conquer in any domain 
    \end{itemize}
    Computational thinking for scientists, engineers, and other professionals further means being able 
    to 
    \begin{itemize}
    \item Apply new computational methods to their problems 
    \item Reformulate problems to be amenable to computational strategies 
    \item Discover new �science� through analysis of large data 
    \item Ask new questions that were not thought of or dared to ask because of scale, easily 
    addressed computationally 
    \item Explain problems and solutions in computational terms
    \end{itemize}
