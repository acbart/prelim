\section{Planned Intervention}

With the development of the new technology, interventions will be staged through Virginia Tech's new ``Introduction to Computational Thinking'' course, created to help fulfill the university's new General Education Requirements~\cite{vt-vision}. 
This course has already been run for two semesters, deeply incorporating much of my existing research work.
The course is taught and developed primarily by Dr. Dennis Kafura, although I have also been involved as associate instructor, managing course materials, server and technology administration, and assisting with in-class teaching.
However, now that the course is more solidly defined, my role is shifting into a more observational function in order to drive my dissertation work.
As a research endeavor, the course is heavily instrumented to provide data on its novel pedagogies and technologies.
Although there are confounding factors to working with such a heavily experimental course, it presents a unique testbed for materials and is an excellent source for mining research results.

\subsection{The Learners}

In the first offering of the course, 25 students enrolled in the course, and 20 students finished the coursework.
In the second offering, 40 students initially enrolled and 35 successfully passed the course.
Figure \ref{data-demographics} indicates the relevant demographic data collected through surveys.
Largely, the students represent the population at Virginia Tech, albeit with some disproportion towards certain majors than others.
It is worth pointing out the excellent gender diversity within the class.

\begin{figure*}
\begin{minipage}{\linewidth}
\centering
	\begin{tabular}{c|c|c}
	  \multicolumn{3}{c}{Gender}\\\hline
		& Fall 2014 & Spring 2015 \\\hline
		Female & 6 & 21 \\
		Male & 13 & 18 \\
	\end{tabular}
\centering
	\begin{tabular}{c|c|c}
	  \multicolumn{3}{c}{Prior Programming Experience}\\\hline
		& Fall 2014 & Spring 2015 \\\hline
		Yes & 10 & 7 \\
		No & 10 & 32 \\
	\end{tabular}
\centering
	\begin{tabular}{c|c|c}
		\multicolumn{3}{c}{Year}\\\hline
		& Fall 2014 & Spring 2015 \\\hline
		Freshman & 2 & 5 \\
		Sophomore & 7 & 11 \\
		Junior & 6 & 11 \\
		Senior & 5 & 10 \\
		Unknown & 0 & 2 \\
	\end{tabular}
\centering	
	\begin{tabular}{c|c|c}
	  \multicolumn{3}{c}{Colleges}\\\hline
		& Fall 2014 & Spring 2015 \\\hline
		Engineering & 2 & 2 \\
		Agriculture & 0 & 1 \\
		Sciences & 7 & 5 \\
		Liberal Arts & 9 & 23 \\
		Architecture & 1 & 7 \\
		Natural Resources & 0 & 0 \\
	\end{tabular}
\caption{Demographic Data of Computational Thinking Students}
\label{data-demographics}
\end{minipage}
\end{figure*}


\subsection{The Content}

Virginia Tech defines four learning objectives for computational thinking. Students are required to:
\begin{enumerate}
	\item Apply computational methods to model and analyze complex or large scale phenomena.
	\item Formulate problems and find solutions using computational or quantitative thinking in their field of study.
	\item Give examples of the application to, and discuss the significance of, computational thinking in at least two different knowledge domains. 
	\item Evaluate the social and political impact of computing and information technologies 
\end{enumerate}

This content is mapped roughly into four instructional units on Computational Modelling, Algorithms, Data Intensive Inquiry, and Social Impacts. The lattermost unit is threaded throughout the course, while the first three are roughly sequential. Figure \ref{course-outline} gives a high-level overview of the content of this course.

\begin{figure*}
\begin{tabularx}{\textwidth}{ |l|X| }
\hline
Topic (Length) &	Description \\\hline
Computational Modeling \newline\newline
 (2 weeks) & Model-based investigation of how complex global behavior arises from the interaction of many “agents”, each operating according to local rules. Students use case-based reasoning and encounter basic computation constructs in a highly supportive simulation environment. \\\hline
Fundamentals of Algorithms \newline (4 weeks) & Study of the basic constructs of programming logic (sequence, decisions, and iteration) and program organization (procedures). A block-based programming language is used to avoid syntactic details. Students can see how these constructs are expressed in Python. \\\hline
Data-intensive Inquiry \newline (7 weeks) & Project-based exploration of complex phenomena by algorithmically manipulating large-scale data from real-world sources. Students construct algorithms in Python using a supportive framework for accessing the data. \\\hline
Social Impacts \newline (2 weeks) & Explore and discuss contemporary societal issues involving computing and information technology. \\\hline
\end{tabularx}
\caption{High-Level Course Overview}
\label{course-outline}
\end{figure*}


\subsection{The Course}

The course uses a considerable amount of modern pedagogical techniques, many of which represent ongoing research questions.
Perhaps the most influential technique is the organization of students into cohorts.
Near the beginning of the semester, students are put into groups of 5-6, balancing based on year and gender where possible, and avoiding putting similar majors into the same group.
These cohorts primarily function as a support structure that students can rely on to get help and encouragement.
Although cohorts work together on many smaller in-class assignments, every student is ultimately responsible for their own work -- the final project, for instance, is individual to each student.

Class time is split between presentation (typically stand-up lecture) and participation (typically computer-based work or cohort discussion) using an Active Learning style whereever possible.
Earlier questions in the course often have students completing questions on paper or doing more kinetic exercises.
Later questions rely on the automated Kennel questions, until the students reach the open-ended project work.

Work in the class is considered to be under a mastery style -- deadlines are enforced only in so far as they motivate students to complete assignments, not in order to punish.
Students are free to work on the material as long as they need.
A recurring message within the course is that ``failing is okay, as long as you keep trying''.

\subsection{The Research Protocol}

The Computational Thinking course is deeply integrated into my research technology -- the middle section of the course uses Kennel as the programming environment, and the last section of the course uses CORGIS to manage the students' projects.
Every semester, students will be surveyed at the start, middle, and end of the course in order to collect relevant data.
Their interactions with the technological systems will be tracked at a fine-grained level, as will be their final projects.
This data can be compared across semesters as it accumulates in order to hit instructor-set goals.
Along the way, important questions related to secondary research goals can also be answered -- how to design CORGIS libraries, the affordances of Data Science as an educational tool, and the pedagogical strategies for using Kennel.

\subsection{Measuring Success}

There are three primary dimensions used in measuring the success of my proposed work: students' motivation, students' ability, and the courses' scalability.
Different instruments and sources will be used to track and measure the impact of the various research efforts.
Ultimately, gains are expected as materials are refined, ideally achieving some established performance objectives.

\subsubsection{Motivation}
Motivation in the course will be primarily assessed through self-reported surveys based heavily on the MUSIC Model of Academic Motivation Inventory (MMAMI).
Students will be non-anonymously surveyed using the complete instrument at key points in the semester, and regularly surveyed using a miniaturized version throughout the semester.
These miniaturized versions will be used as follow-ups to specific course exercises in order to determine more finely-grained answers about where students found motivation.
Additionally, targeted qualitative data will be gathered in order to refine this quantitative data.

This motivational data will be combined with evidence of engagement with the course across several different dimensions. Therefore, the complete list of motivational sources is:
\begin{description}
	\item[Attendance:] Did the students regularly attend class?
	\item[Retention:] Did the students complete the course with a passing grade?
	\item[Coursework:] Did the students regularly complete coursework and homework on time?
	\item[Struggle:] Did the students continue with materials even after setbacks?
	\item[Continuation:] Do students intend to follow-up the course with further courses and learning experiences in Computational Thinking?
	\item[Observations:] Did the course staff report students as being motivated, according to their own observations?
	\item[Self-report:] Did the students report themselves as being motivated, according to MMAMI and other surveys?
\end{description}

\subsubsection{Ability}
There are several high-level course concepts that students are expected to gain some level of mastery in during this course.
There are also lower-level skills that students should gain some familiarity with.
Although this is not a programming course, per se, success will be partially measured based on students success with the computational elements of the material.
However, the success of the course will be foremost evaluated on the high-level concepts, such as the role of Abstraction and Algorithms.

The most important element measuring students' ability is the final project, a performance assessment where students solve a self-guided, computationally-oriented problem that they might realistically find in their chosen profession.
Students choose their own dataset (through CORGIS), problems, and approach based on the techniques learned in the course and supported by the course staff.
Course staff are ultimately responsible for evaluating their completed project (instantiated as a codebase and a video presentation) against a rubric that measures several of the higher learning objectives.
This form of assessment is viewed as more authentic than alternatives such as a final exam or a concept inventory, given the complicated nature of measuring high-level conceptual skills such as Computational Thinking.

Smaller, low-stakes, pre- and post- assessments will also be collected during the semester for individual instructional units in order to gather data on students’ ability with specific, lower-level course content (e.g. programming concepts such as loop iteration and dictionary access).
This data will be supplemented by fine-grained, automatically-collected student solutions to coding problems in order to evaluate student progress throughout their interactions with an assignment, rather than purely summatively.
Finally, course staff will be regularly and systematically interviewed in order to identify the commonly-occurring questions and challenges that students raise.
Ultimately, this data should approach some specific measurement of success -- this specific measure will be determined shortly before the preliminary proposal based on data collected during the first two course offerings.
As a rough measure, 70\% of the students should achieve a ``high'' level of understanding of all the course concepts, and no more than 15\% of the students should achieve a ``low'' or worse level of understanding, as measured by the instructors.

\subsubsection{Scalability}
As a major stakeholder in the course, the instructors’ experience is also crucial to understanding the success of our approach.
As the course scales in size, it is both expedient and proper that an instructor have a feel for their learners.
The technology in the course must support this process wherever possible – for example, an instructor meeting with a student during office hours must immediately know the students current state in the course, both motivationally and ability-wise.
Course staff will be regularly interviewed in order to identify successes and failures of the automated tools in managing a large-scale course.
Students will also be interviewed and surveyed to assess their perception of the courses size – e.g., whether feedback came timely enough, they had the in-class support they needed to succeed, they felt satisfied with the human interactions in the course.

