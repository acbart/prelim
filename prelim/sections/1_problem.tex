\section{Problem Statement}

Computing is increasingly considered a 21st century competency, leading to its growing entrenchment within universities’ general education curriculum~\cite{wing2006}.
At Virginia Tech, for instance, the core requirements at the university have recently shifted to require all undergraduate students to take credit hours in ``Computational Thinking'' -- emphasizing crucial topics in Abstraction, Algorithms, Computing Ethics, and more.
The students in this course represent a diverse spread of majors from the arts, the sciences, engineering, agriculture, and more.

Although they have rich backgrounds, non-major learners are a challenge for instructors because they have no prior background in computer science, are not confident about their ability to succeed in the course, and have no assurances that it will be a useful experience.
These students are at-risk for dangerously low levels of motivation.
As a more mature audience with domain-identified career goals and needs, existing interest-driven approaches to introductory computing may not be sufficiently effective, because students will not be convinced to learn material that doesn't have long-term benefit~\cite{guzdial2006imagineering}.
How do we motivate these unique learners?

This work will develop new pedagogical approaches and technical tools to better understand and resolve this question.
In particular, I will explore the efficacy of Student-driven Data Science as an introductory context.
This context moves beyond basic models of motivation to take advantage of multiple dimensions of engagement.
Using complicated, real-world datasets is a known technique used in machine learning courses, and has even seen use in non-CS graduate level courses using the Software Carpentry method~\cite{Wilson-SoftwareCarpentry}.
However, I propose new technology and pedagogy to enable its use in the introductory level.
Further, I will conduct more comprehensive and fruitful evaluations than exists in the current literature.
