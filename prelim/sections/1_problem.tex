\section{Problem Statement}

Computational Thinking is increasingly considered a 21st century competency, paralleling its growing entrenchment within universities’ general education curriculum~\cite{wing2006}.
Although the term itself is historically ill-defined, most definitions consider it the application of computer science concepts and computational techniques to frame problems and devise solutions across disciplines~\cite{weinberg2013}. 
Although Computational Thinking is more than just programming, programming is a key element of Computational Thinking -- and introducing programming is an on-going struggle within the field of Computer Science education.
This struggle is magnified when brought before a more general public, the primary goal of the Computational Thinking movement.

At Virginia Tech, the core requirements at the university have recently shifted to require all undergraduate students to take credit hours in Computational Thinking.
These students represent a diverse spread of majors from the arts, the sciences, engineering, agriculture, and more.
Non-major learners represent a challenge because they have no prior background in computer science, have no assurances that it will be a useful experience, and are not confident about their ability to succeed in the course.
Further complicating these problems is that enrollment for such courses must scale aggressively due to their critical role within new general education requirements -- few departments have the expertise and resources necessary to effectively teach the subject.
At the same time, an active, collaborative learning experience is optimal for meaningfully imparting the course material – lecture should be minimal, and class time should be a chance to work with the material, interact with classmates, and receive support from the course staff.
This primary problem is therefore composed of three pedagogical research questions:
\begin{enumerate}
	\item How do we motivate these unique learners?
	\item How do we guide these learners to achieve success?
	\item How do we scale the instructional materials to as large a population as possible?
\end{enumerate}

This work will develop new pedagogical approaches and technical tools to better understand and resolve these questions.
In particular, we will explore the efficacy of Big Data Science as an introductory context and efficacy of a scaffolded Block-based Programming Environment that utilizes advanced techniques in program analysis and software engineering to guide the learner.

