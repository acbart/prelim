\section{Literature Review}



\subsection{Introductory Computing Content}

There is a serious, on-going debate about what novice students in computing should learn, with limited consensus. 
Complicating this discussion is the bifurcation of undergraduate introductory computing into Computational Thinking (sometimes referred to as CS-0) and Computer Science (sometimes referred to as CS-1), and the entirely different curriculum proposed for K-12 education.
There are some commonly agreed aspects however: any good introductory computing course will cover abstraction (representing complex phenomena more simply, usually as coded data), some form of decision-making (e.g., \texttt{if} statements, \texttt{cond} statements), and some form of iteration (e.g., \texttt{for} loops, recursion)~\cite{Kramer:2007, CS2013, csta-computational-thinking}.
From there, different curriculums lay out the material differently.
The ``How to Design Programs'' curriculum emphasizes a functional programming model, using a LISP-descended language named Racket, and the only looping mechanism that students are taught is recursion.
The dominance of the Object-Oriented Model in software engineering usually leads to a strong emphasis in introductory courses on abstracting data using Objects and Classes -- in fact, there is even an ``Objects First'' curriculum.
Of course, besides the technical aspects, definitions include softer skills such as a tolerance for unstructured problems and collaborative attitudes~\cite{csta-computational-thinking, google-computational-thinking}.

In general, this proposal will not take a strong view on what should be included in an introductory experience, beyond a core of Abstraction and Algorithms.
However, for practical purposes, materials and research work are grounded and tested in a course, and have adaptions for that content.
Specifically, this proposal is used in a Computational Thinking course, so that is how the content is oriented.
``Computational Thinking'' was coined by Seymour Papert in 1993~\cite{papert1996} and popularized by Dr. Jeannette Wing's 2006 paper~\cite{wing2006}, which opened a floodgate of discussion about the term. 
Unfortunately, there is still limited consensus on \textit{what} exactly CT is, whether it should be universally taught, how it should be taught, and how to identify when it has been taught.

An excellent resource for summarizing the history of Computational Thinking research is the 2013 dissertation by Wienberg~\cite{weinberg2013}. 
This comprehensive survey analyzed 6906 papers directly or indirectly related to Computational Thinking from 2006-2011, describing research efforts and findings.     
Over half of the research on CT describes approaches to pedagogy (Curriculum and Program Description), leaving a small amount to modeling (Philosophy and Opinion) and assessment (Research and Evaluation). 
The lack of assessment research is understandable given the youth of this area of research, but still troubling.
Even more troubling, however, is the further analysis of the 57 empirical studies.
Only fifteen (26\%) studies include or sought an operational definition of computational thinking, and only six go beyond the superficial (solely describing computational thinking as a "way of thinking", a "fundamental skill", or a "way of solving problems"). 
The failure to identify an operational definition weakens the theoretical strength of the studies.
This weakness likely stems from the background of the researchers: 28\% of the articles involved non-CS majors and \textbf{only 18\% of the articles involved education experts.} 
In other words, over four-fifths of this educationally-oriented research was performed by people with no real formal training in educational research techniques.
This is particularly troubling given that Computational Thinking is a strong target for interdisciplinary endevaors.

Weinberg reflects on the continuing debate about the importance of Computational Thinking:
\begin{quote}
    Many, like Wing, believe computational thinking to be a revolutionary concept, one as 
important to a solid educational foundation as are reading, writing, and arithmetic (Bundy, 2007\cite{bundy2007}) 
(Day, 2011\cite{day2011}). Others believe its potential and significance are overstated (Denning, 2009\cite{denning2009}; 
Hemmendinger, 2010\cite{hemmendinger2010}), and some have voiced concern that by joining forces with other 
disciplines computer science might be diluting either one or both of the participating disciplines 
(Cassel, 2011\cite{cassel2011}; Jacobs, 2009\cite{jacobs2009}). Both the praise and the criticism for computational thinking could 
perhaps be tempered by reflecting on a historical quote by Pfeiffer in 1962: “Computers are too 
important to overrate or underrate. There is no real point in sensationalizing or exaggerating 
activities which are striking enough without embellishment. There is no point in belittling 
either.” (Pfeiffer, 1962\cite{pfeiffer1962}).
\end{quote}

Although it is ambiguous what Computational Thinking is, we will take it as a given that it requires students to learn some amount of non-trivial programming.

\subsection{Introductory Computing Contexts}

As part of the overarching goal to bring more students into Computer Science, a large number of contexts have been explored in Introductory computing. 
The context of a learning experience grounds the learner in what they already known, in order to teach the new material.
Many introductory computing experiences focused on presenting the content as purely as possible, which can come across as abstract and detached~\cite{Zografski}.
However, starting with Seymour Papert's work with robotics and the LOGO programming environment in the 70s~\cite{papert1996}, instructors have been interested in motivating students' first experience with richer contexts.
Some of these contexts rely on Situational Interest (e.g.,  Digital Media ``Computation'' (Manipulation)~\cite{Forte} and Game Design~\cite{Zografski}), while others attempt to provide enduring career value (e.g., [Big] Data Science ~\cite{Anderson}) or short-term social applicability (e.g,. Problem Solving for Social Good~\cite{SocialGoodinComputingEducation}).
Ultimately, each of these approaches draws on different facets of motivation -- they do not partition the space, but reside in it organically depending on their implementation.
In this section, I will discuss the implications of these different approaches.


\subsubsection{Abstract Contexts}

Denning describes the early perception of Computer Science by the public as ``stodgy and nerdy''~\cite{Denning:2005}, since many early computer science classes were driven so strongly by mathematics and logic.
A common early introductory programming problem, for instance, is writing a function to compute a fibonacci number -- a relatively simple task if you are familiar with the recurrence, and one that leads quite nicely to discussions on the implementation of algorithms, computational complexity, and a host of other subjects~\cite{crazypantsfibonaccipaper}.
These contexts are ``abstract'' because they are already at a similar level of abstraction as the content they are attempting to convey.
However, Oliveira~\cite{ConcreteVsAbstract} suggests that the discussion about ``abstract vs. concrete'' contexts is a misleading one, because the purity with relation to the content is less important than \textit{prior knowledge}.
According to modern constructivist and cognitivist theories, learners build on prior knowledge, and the ability to relate to what they know is crucial.
The simple fact is that most students are not particularly good at mathematics, so relying on it as a context is not a useful approach compared to finding subjects that students know and understand readily.

\subsubsection{Situationally Interesting Contexts}

As it became clear that Computer Science had a serious image problem, work began on making Computer Science ``Fun'' and approachable. 
A key goal was to increase diversity and to broaden participation. 
This led to the rise of Situationally Interesting Contexts, emphasizing problems and projects that would be immediately appealing to a wide audience.
Guzdial, for instance, was largely responsible for the creation of the Media Computation approach, where students use computational techniques (e.g., iteration and decision) to manipulate sound, images, videos, and other digital artifacts.
As an example, students might use a nested, numerically-indexed \texttt{for} loop in order to adjust the red-value of the pixels in an image, treating it as a two-dimensional array of binary tuples, in order to reduce the red-eye of a photo.

Although wildly deployed, a review of these curricular materials by Guzdial \cite{guzdial2006imagineering} in light of Situated Learning Theory found that students did not find this an authentic context, and intense rhetoric was insufficient to convince them that it was authentic. 
Few students find it expedient and helpful to remove the red-eye from family photos by writing python scripts, when they could easily use a GUI-based program to automate the task instead.
Guzdial leaves open the question of what contexts can be truly authentic for non-majors, given the relative novelty of teaching introductory computing for non-majors.
Ben-Ari echos this question by suggesting a very narrow selection of authentic contexts and communities in his paper exploring the application of Situated Learning Theory to Computer Science in general \cite{ben2004situated}.
Critically, the opposite problem could occur -- if an instructor is effective at convincing students a context is authentic, they may believe them.
There are serious ethical issues involved in mispresenting the utility of a context, leading students to develop an embarrassing misconception of the field -- imagine a young child believing that all of Computer Science is game design, because that is what they started off doing.

There are other disadvantages of an Interest-driven approach.
The motivation literature describes ``Seductive Details'' (interesting but irrelevant adjuncts)~\cite{harp1998seductive} as interfering both with short-term problem completion and long-term transfer.
In other words, students get hung up on unimportant aspects of the context that they ignore the content.
Consider a student using the game and animation development environment Scratch, which allows beginners to create sprites from images.
A young learner may be so amused by the ability to change the color and shape of their image, that they neglect their assigned work.
Although a well-regulated learner would not be distracted, most of the at-risk population that would benefit from these contexts are unable to deal with such distractions.
Of course, this could be said of any context, but there is particular danger from this kind of context.

Kay~\cite{Kay:2011} identifies another, potentially critical problem of relying soley on situationally interesting contexts, particularly when it leads to individualized interest towards that context and not the content. 
What happens once a student has completed the introductory course and is ready to move onto further course?
Most later courses are more decontextualized, and will not use contexts such as game development, robots, etc.
Kay goes so far to say that it is unethical to suggest to students that a contextualized introductory course is representative of the curriculum as a whole.

\subsubsection{Empowered Contexts}

Orthogonal to the idea of making a context fun is the idea of giving students more freedom and agency to control their learning.
Compared to Interest-driven contexts, this approach is comparatively less researched, although it is not an uncommon practice -- instructors will often allow students to choose from a range of projects or assignments.
Stone~\cite{EmpowermentInProjects} ran a 2-year study where students were allowed to choose their projects from a wide range of domain areas (e.g., Biology, Math, Business, Etymology), and were then surveyed on their engagement.
Unfortunately, their experiment suffered strongly from low enrollments and even lower survey responses; it is difficult to believe any of their results (including the idea that women are more likely to prefer biological- and meterological-themed projects).
However, their experiences do suggest an interesting challenge: normalizing the difficulty (both in terms of computational knowledge and domain knowledge) across many different projects is a struggle.

\subsubsection{Contexts that Make Instructors Care}

Most modern educational theories argue that learning is inescapably affected by social factors.
There is evidence that the instructor~\cite{thompson2009engine} and fellow students~\cite{Barker:2009} are the most important factors in an introductory experience, for instance.
Kay~\cite{Kay:2011} discusses this explicitly.
It is possible that the most important element in a context is not whether it is fun or useful, but whether the instructor can get excited about it and impart that enthusiasm to the student.
And not just enthusiasm, but a thorough understanding of the problem, its usefulness, and the rest of its attributes.

\subsubsection{Useful-Driven Contexts}

An alternative focus to Interest is Usefulness, the idea that the context should have immediate or eventual benefit to the learner's needs.
To some extent, it is impossible (or at least prohibitively difficult) to find a one-size-fits-all context that will be useful to all learners (doing so is probably an example of preauthentication~\cite{preauthentication}, or designing learning experiences without your learners in mind).
The ideal situation for any instructor is to create contexts that specifically suit the interests and values of your learners~\cite{DiSalvo:2011}.
However, in practice, some contexts are broadly useful and are likely to engage a diverse crowd of learners.
In this section, I suggest two distinct contexts that might fall into this role: real-world problem solving and data science.

In theory, Computer Science provides tools for solving problems, and it is possible that the problem solving can be done with even the simplest tools~\cite{Layman:2007, Social-good}.
Many authors collaborated to produce a new framework centered around these ideas -- ``Social Computing for Good'', a collection of approaches and projects for interdisciplinary students to solve using computing ~\cite{Social-good}.
They raise a number of issues with using socially relevant materials: that games and graphics can be more appeal to instructors as a ``cheap'' source of motivation, that students and instructors can become cognitively overloaded by the addition of domain knowledge, and that instructors can even be intimidated if they don't have expertise in the domain area.
They also create a valuable rubric for developing and evaluating problems (which I map to elements of the MUSIC model below):
\begin{enumerate}
	\item The degree to which the problem is student-directed (eMpowerment)
	\item The amount of scaffolding needed (Success)
	\item The amount of external domain knowledge needed (Success)
	\item The contribution to the Social Good (Usefulness, Caring)
	\item The ``coolness'' or ``sexiness'' (Interest)
	\item The amount of explicit student reflection incorporated (Usefulness)
\end{enumerate}
Although this framework presents some ideas, there are still unsolved technical and pedagogical problems in how to optimally bring these materials to learners; their paper ends by raising questions about the effectiveness of this approach compared to existing methods.

A number of other researchers have created course materials, with varying degrees of evaluation.
Erkan~\cite{Erkan:2012} had a sustainability themed curriculum -- student surveys completed afterwards suggested some level of effectiveness, although the sample size (N=16) was far too small.
In addition to providing two case studies, Buckley ~\cite{Buckley:2008} suggests an interesting delineation between Social problems (uppercase S, indicating problems general to society) and social problems (lowercase S, indicating problems of personal interest to the learner).

Barker ran a large survey asking why students persist towards majoring in CS~\cite{Barker:2009} (N=113, only freshmen).
Critically, they found that Meaningful/Relevant Assignments (subsuming both Interest and Social Usefulness) was a major factor in whether students would persist in the major -- however, it wasn't one of the top ones.
Interestingly, whether students felt that their workload and pace was appropriate was a much bigger source of importance, suggesting that care and attention should be given to making a context suitably difficult before it is made interesting and useful.
This focus on ensuring normalized, appropriate difficulty is echoed by several other authors, all of whom suggest that doing so is not trivial~\cite{Rader:2011, Stevenson:2006}.

In the past two decades, the field of Data Science has emerged at the intersection of Computer Science, Statistics, Mathematics, and a number of other fields.
This field is concerned with answering real-world problems through data abstractions, and offer a less socially-conscious path to Usefulness.
As a context, there are pedagogical penalties for using it, since it introduces a wide variety of new content -- visualization, statistics, ethics, social impacts... the list is long~\cite{Anderson:2015-DP2}.
However, a good instructor can downplay the focus on these side-areas as needed, or even emphasize subject matter's strengths (e.g., a statistics major might find it interesting to use their mathematical background to strengthen their problem-solving investigation).
However, there are other difficulties: bringing in messy data requires real sophistication by the instructor, especially when working with Big Data.

The use of data analysis as a form of contextualization is not novel, and represents a new and actively growing movement where instructors create programming assignments with specific datasets in mind ~\cite{Anderson, Sullivan:2013, Hall-Holt:2015, DePasquale:2006}.
Upper division courses have employed these situated learning experiences using data of varying size and complexity for several years \cite{Egger, datamining, Waldman}.
However, in all of these research papers, there is typically very little evaluation of the advantages and disadvantages of using data science in introductory education.
Although Sullivan ~\cite{Sullivan:2013} did conduct a study on the difficulty and usefulness of the datasets they provided, they do not go very far in identifying lasting lessons for educators creating such datasets.
Other researchers conducted even less impressive studies: DePasquale~\cite{DePasquale:2006} included exactly ONE student response in their evaluation.
